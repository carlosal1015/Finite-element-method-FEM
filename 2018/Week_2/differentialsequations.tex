\documentclass{memoir}
\usepackage{amsmath,amssymb,amsthm}

\begin{document}
\chapter{Aproximación numérica y errores}


\section{Error absoluto}

\[ e_a=|x-x_a| \]

\section{Error relativo}

\[e_r=|\frac{x-x_a}{x}| \]

\section{Concepto de convergencia y estabilidad de un método iterativo}

\section{Método de aproximaciones sucesivas}

Parten de una aproximación inicial $x_0$ a la solución $x$ del problema, y mediante la aplicación reiterada de una o varias fórmulas de recurrencia proporcionan aproximaciones $x_1$, $x_2$, $x_3$, \ldots , $x_n$ a la solución de $x$.


\section{Método de paso a paso}

Son aquellas que parten de un valor inicial y se basan también en la aplicación de una fórmula de recurrencia; pero a diferencia de los métodos de aproximaciones sucesivas, se utilizan para obtener aproximaciones a la solución de una sucesión de números, en lugar de un solo valor.


Los métodos iterativos no siempre proporcionan aproximaciones aceptables y, en muchos casos, el error que se obtiene al aplicarlos aumenta a medida que se incrementa el número de iteraciones. Dos de las principales causas del aumento del error se explican con los conceptos de \textit{convergencia} y \textit{estabilidad} que a continuación se definen.


\section{Convergencia}

Se demuestra que si

\[ |x_n - X_{n-1}|\ge e_a \]

\section{Estabilidad}
Un sistema es \emph{estable} si para pequeñas variaciones en la entrada, se produce pequeñas variaciones en la salida.

Sea $E(n)$ la función error, que representa el error absoluto en la salida del algoritmo, después de $n$ iteraciones.

Si $E(n)$ se incrementa de modo lineal conforme $n$ aumenta, se dirá que el método es \textit{estable}.

Si $E(n)$ aumenta en forma exponencial o aproximadamente exponencial, se dirá que el método es \textit{inestable}.


Vamos a comparar los métodos en base a su eficiencia.

Se dice que el método $A$ es más eficiente que el método $B$ si en general, con $A$ se realizan menos operaciones que con $B$.

Un método se considera más eficiente que otro si en general requiere de menos instrucciones para programarse o utiliza menos cantidad de memoria.

\chapter{Solución numerica de ecuaciones algebraicas y trascendentes}

\section{Método de la bisección}

\section{Método del punto fijo}




\end{document}