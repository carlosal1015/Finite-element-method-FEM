\usepackage{xcolor}
\usepackage{ngerman}
\usepackage{lscape,dsfont}

\usepackage{eulervm}
\usepackage{DejaVuSans}
\usepackage{tgpagella}
\usepackage[T1]{fontenc}


\usepackage[utf8]{inputenc}% statt latin1 evtl. utf8 probieren

\usepackage{amsthm}
\usepackage{amsmath}
\usepackage{amssymb}
\usepackage{graphicx}
\usepackage{booktabs}

\setlength\parindent{0pt} 

\newcommand{\Oc}{\mathcal{O}}
\newcommand{\oc}{\mathrm{o}}

\newcommand{\sk}{\bigskip\hrule\bigskip}


\newcommand{\LL}{\bigskip ~\hrule ~ \bigskip}

\newcommand{\UB}[2]{%
\noindent
Malte Braack, University of Kiel \hfill #2\\
Carolin Mehlmann, University of Magdeburg\\
Thomas Richter, University of Magdeburg\\


\begin{center}
  \textbf{Exercise Nr. #1, \emph{Summer School on Finite Elements}}\\
  \textbf{Universidad Nacional Agraria La Molina}\\  
  \textbf{December, 2018}
\end{center}
\vspace{0.25cm}
}

\newcommand{\PR}[2]{%
\noindent
Institut f"ur Analysis und Numerik \hfill #2\\
Universit"at Magdeburg\\
Thomas Richter

\begin{center}
  \textbf{Programmierprojekt Nr. #1 zur Vorlesung Algorithmische Mathematik I}\\
  \textbf{Wintersemester 2018/2019}
\end{center}
\vspace{0.25cm}
}

\newcommand{\PraesenzUB}[2]{%
\noindent
Institut f"ur Analysis und Numerik \hfill #2\\
Universit"at Magdeburg\\
Thomas Richter


\begin{center}
  \textbf{Pr"asenz"ubung Nr. #1 "=  Einf"uhrung in die
    Numerik}\\  
  \textbf{Wintersemester 2015/16}
\end{center}
\vspace{0.25cm}
}


\newcommand{\LO}[2]{%
\noindent
Institut f"ur Analysis und Numerik \hfill #2\\
Universit"at Magdeburg\\
Thomas Richter

\begin{center}
\textbf{L"osung zu: "Ubung Nr. #1 zur Algorithmische Mathematik 2}\\
\textbf{Sommersemester 2017}
\end{center}
\vspace{0.25cm}
}


\newcommand{\Projekt}[2]{%
  \bigskip
  \noindent{\textbf{Projekt #1 -  #2} }
  \medskip
}


\newcommand{\ZusatzAufgabe}[2]{%
  \bigskip
  \noindent{\textbf{ Zusatzaufgabe #1: (#2 Bonuspunkte)} }
  \medskip
}

\newcommand{\Aufgabe}[2]{%
  \bigskip
  \noindent{\textbf{ Aufgabe #1: (#2 Punkte)} }
  \medskip
}
\newcommand{\ProgrammierAufgabe}[2]{%
  \bigskip
  \noindent{\textbf{ Programmieraufgabe #1: (#2 Punkte)} }
  \medskip
}


\newcommand{\Loesung}[1]{%
  \bigskip
  \hrule
  \bigskip
  \noindent{\textbf{ L"osung #1:} (5 Punkte) }
  \medskip
}

\newcommand{\LoesungPraesenz}[1]{%
  \bigskip
  \hrule
  \bigskip
  \noindent{\textbf{ L"osung Pr"asenzaufgabe #1:} (5 Punkte) }
  \medskip
}

\newcommand{\LoesungPunkte}[2]{%
  \bigskip
  \hrule
  \bigskip
  \noindent{\textbf{ L"osung #1:} (#2 Punkte) }
  \medskip
}

\newcommand{\PAufgabe}[2]{%
  \bigskip
  \noindent\textbf{ Praktische Aufgabe #1: (#2 Punkte) }
  \medskip
}

\newcommand{\Abgabe}[1]{%
  \bigskip
  \hrule
  \bigskip

  \noindent{\textbf{ Abgabe}} der "Ubungen bis {\textbf{ #1}} im
  Kasten vor Raum 5, Geb"aude 2 oder vorher per Mail an
  \texttt{algomath@ovgu.de}.  

%  \noindent Ab Blatt 6 Abgabe der theoretischen "Ubungen in
%  \textbf{festen Zweiergruppen m"oglich.} Die Zweierteams müssen dabei
%  fest sein und nicht von Blatt zu Blatt wechseln. 

  \medskip
  Abgabe der praktischen Programmier"ubungen (weiterhin Einzelabgabe)
  bis {\textbf{#1}} per   Mail an  
  \texttt{algomath@ovgu.de}.    Den \textbf{Betreff} der Mail bitte
  mit Name, Nummer des Blattes, z.B.\\
  \texttt{Blatt 2 - Monika Mustermann}


  \medskip
  Den \textbf{Betreff} der Mail bitte mit Name, Nummer des Blattes,
  z.B.\\
  \texttt{Blatt 2 - Monika Mustermann - Programmieraufgabe}
  
  \medskip
  Programme k"onnen einfach in die Mail kopiert werden und müssen
  nicht als Anhang gesendet werden.
}

\newcommand{\AbgabeProjekt}[1]{%
  \bigskip
  \hrule
  \bigskip

  \noindent{\textbf{ Abgabe}} des Programmierprojekt bis {\textbf{
      #1}}. Abgabe eines Programms per Mail an
  \texttt{algomath@ovgu.de}, Abgabe der (kurzen) Auswertung in den
  Briefkasten vor Raum 5, Geb"aude 2 - oder ebenso per Mail an
  \texttt{algomath@ovgu.de}.  

  \medskip
  Den \textbf{Betreff} der Mail bitte mit Name, Nummer des Projektes
  und Gruppe.

  \medskip
  Programme k"onnen einfach in die Mail kopiert werden und müssen
  nicht als Anhang gesendet werden.

  \medskip
  \noindent Die Programmierprojekte sollen in einer 3er oder 4er
  Gruppe bearbeitet werden. In den "Ubungsgruppen am 07. und
  08. Januar werden die Projekte besprochen. Alle Teilnehmer der
  Gruppe sollten nat"urlich an den Programmen mitgearbeitet haben und
  einfache Fragen zu den Programmen beantworten k"onnen. 
}

\newcommand{\PraesenzAbgabe}[1]{%
  \bigskip
  \hrule

  \noindent{\textbf{ Keine Abgabe}} der "Ubungen. Besprechung im Tutorium
  der  {\textbf{ #1}}.\\
  \medskip
}


\newcommand{\file}[1]{%
  \noindent \textsc{Datei:} \texttt{#1}

  \smallskip
}

\newcommand{\code}[1]{%
  \texttt{#1}}

\newcommand{\bigcode}[1]{%
  \texttt{#1}
}

%%% Local Variables: 
%%% mode: latex
%%% TeX-master: t
%%% End: 
