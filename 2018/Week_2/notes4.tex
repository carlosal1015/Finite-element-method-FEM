% arara: clean: {
% arara: --> extensions:
% arara: --> ['log','aux','synctex.gz','out']
% arara: --> }
% arara: lualatex: { draft: yes }
% arara: lualatex: {
% arara: --> shell: yes,
% arara: --> synctex: yes,
% arara: --> interaction: batchmode
% arara: --> }
% arara: clean: {
% arara: --> extensions:
% arara: --> ['log','aux','synctex.gz','out']
% arara: --> }
\documentclass[
	a4paper,
	11pt,
	oneside
]{scrreprt}

\usepackage{mathpazo}
\usepackage{fontspec}
\setmainfont{TeX Gyre Pagella}
\usepackage{amsmath,amssymb,amsthm,bm}
\theoremstyle{definition}
\newtheorem{definition}{Definition}
\newtheorem{theorem}{Theorem}
\renewcommand{\div}{\operatorname{div}}

\usepackage{siunitx}

\newcommand\restr[2]{{% we make the whole thing an ordinary symbol
		\left.\kern-\nulldelimiterspace % automatically resize the bar with \right
		#1 % the function
		\vphantom{\big|} % pretend it's a little taller at normal size
		\right|_{#2} % this is the delimiter
}}

\usepackage[
	colorlinks,
	citecolor=blue,
	linkcolor=red,
	pdfpagelabels
]{hyperref}

\hypersetup{pageanchor=false}

\begin{document}

\section{The Navier-Stokes equation}

\[ \rho\left(\partial_{t}v+\left(v\cdot\nabla v\right)\right)-\div=\rho f\quad\div v=0. \]

\begin{itemize}
	\item This term describes the inertia of the fluid in motion.
	\item It takes a force to influence.%TODO:Falta agregar el signo mas en la ecuación.
\end{itemize}

\begin{itemize}
	\item This term describes the \textit{incompresibility} of the fluid, it is not possible to change the volume.
	\item Not every fluid is incompressible, water is incompressible buy air can be compressed if the force is big.
	\item Meaning of incompresiblility. If $V$ is a vvolume then it holds, inflow to $V$ equals to outflow from $V$.
\end{itemize}

\begin{itemize}
	\item The term $\div\sigma$ describes the internal forces within the fluid. (friction)
	\item The Navier--Stokes stress tensor is given by \[ \sigma=\rho\nu\left(\nabla v+\nabla v^{T}\right)-pI\] %TODO: \rho es kinematic viscosity.
	\item $\nu$ is the viscosity of the fluid.
	\item $\rho$ is the pressure.
\end{itemize}

\begin{itemize}
	\item We skip the \textit{inert term} \[ -\nu\Delta v+\nabla p=\rho f\quad\div v=0. \]%TODO: Simply form , no derivative no transportation.
	\item In $2D$: \[ -\nu\Delta v^{1}+ \] $3d$ exists 4 equations.
\end{itemize}

\[ -\nu\Delta v+\nabla p=\rho f\quad\div v=0. \]

\begin{itemize}
	\item Multiplication with test functions $\phi=\left(\phi^{1},\phi^{2}\right)$ and $\xi$ and integration \[ \int_{\Omega}\left(-\nu\Delta v^{1}+\partial_{t}p\right)\phi^{1}\mathrm{d}x=\int_{\Omega}\rho f^{1}\phi^{1}\mathrm{d}x. \]%TODO: In laplace situation belongs to stokes situation.
\end{itemize}

\begin{itemize}
\item The theory of the Stokes equations is very difficult (compared to Lplace)% Laplace is the second order , is elliptic, is positive defined, is non simetry below equation.
\item But, we can simplift the problem

If $(v,p)$ is a solution to the Stokes equation its divergence is zero. \[ \div v=0. \]

We define the space of divergence free functions \[ \mathcal{V}_{h}=\left\{\phi\in H^{0}_{1}\left(\Omega\right)^{2}\mid\div\varphi=0\right\}. \]

Every solution $v$ in this space $v\in\mathcal{V}_{h}$.

Now, we restrict the variational formulation to this space \[ v\in\mathcal{V}_{0}\colon\quad\left(\nu\nabla v,\nabla\phi\right)=\left(\rho f,\phi\right)\forall\phi\in. \]
\end{itemize}

\begin{itemize}
	\item It we have the velocity, the pressure is defined by the problem \[p\in L^{2}\left(\Omega\right)\quad\left(p,\div\phi\right)=\left(\nu\nabla b,\nabla\phi\right)-\left(\rho f,\phi\right)\quad\forall\phi\in H^{1}_{0}\left(\Omega\right)^{2}. \]
	\begin{theorem}
		There exists a unique velocity $v\in\mathcal{V}_0$ and a unique pressure $p\in L^{2}\left(\Omega\right)$ and $\inf-\sup$ \textit{condition} holds \[ \inf_{p\in L^{2}\left(\Omega\right)}\sup_{v\in H^{1}_{0}\left(\Omega\right)^{2}}\frac{\left(p,\div v\right)}{\|p\|\|\nabla v\|}\ge\gamma \] with the inf--sup constant $\gamma>0$.
	\end{theorem}
\end{itemize}

\begin{itemize}
	\item We discretize the variant.
\end{itemize}

\begin{itemize}
	\item Stationary Navier Stokes equations \[ \rho v\cdot\nabla v-\rho\nu\Delta v+\nabla\rho=\rho f\quad\div v=0. \]
	\item Divide by.
\end{itemize}

\section{Reynolds number}

\[ R=\frac{VL}{\gamma} \] where $V$ is the velocity, for example: \[ 100=\frac{v\cdot\SI{10}{\meter}}{\num{e-6}}=\num{1e-5}\cdot V \] $\iff V=\SI{e-7}{\metre\per\second}$.

\[ \operatorname{Re}=\frac{V\cdot L}{\nu}. \]

\begin{itemize}
	\item Length of submarine \[ L=\SI{100}{\metre}. \]
	\item Velocity of submarine \[ V=\frac{\SI{10}{\metre}}s. \]
	\item Viscosity of eater \[ \nu=\frac{\SI{1.2e-6}{\metre\square}}{s}. \]
	\item Reynolds number \[ \operatorname{Re}=\frac{100\cdot 10}{\num{1.2e-6}}\approx833000000. \]
\end{itemize}

\begin{itemize}
\item If the Reynolds number is large we nned very fine meshes \[ h<\sqrt{\frac{1}{\mathrm{Re}}}. \]
\item This is not possible in reality: \[ h<\sqrt{\frac{1}{833000000}}\approx 0.0000035. \]
\item If the domain is $\Omega={\left(\frac{1}{0.000035}\right)}^{2}\approx 24041828902976$ elements ($M=28 000$).
\item This is too much, We must stabilize:
	\begin{itemize}
		\item Artificial diffusion? Stable but too much diffusion.
		\item Streamline diffusion? Stable and good accuracy.
	\end{itemize}
\end{itemize}
\end{document}