% arara: clean: {
% arara: --> extensions:
% arara: --> ['log','aux','synctex.gz','out']
% arara: --> }
% arara: lualatex: { draft: yes }
% arara: lualatex: {
% arara: --> shell: yes,
% arara: --> synctex: yes,
% arara: --> interaction: batchmode
% arara: --> }
% arara: clean: {
% arara: --> extensions:
% arara: --> ['log','aux','synctex.gz','out']
% arara: --> }
\documentclass[
	a4paper,
	11pt,
	oneside
]{scrreprt}

\usepackage{mathpazo}
\usepackage{fontspec}
\setmainfont{TeX Gyre Pagella}
\usepackage{amsmath,amssymb,amsthm,bm}

\theoremstyle{definition}
\newtheorem{definition}{Definition}
\newtheorem{theorem}{Theorem}

\newcommand\restr[2]{{% we make the whole thing an ordinary symbol
		\left.\kern-\nulldelimiterspace % automatically resize the bar with \right
		#1 % the function
		\vphantom{\big|} % pretend it's a little taller at normal size
		\right|_{#2} % this is the delimiter
}}

\usepackage[
	colorlinks,
	citecolor=blue,
	linkcolor=red,
	pdfpagelabels
]{hyperref}

\hypersetup{pageanchor=false}

\begin{document}

\chapter{Laplace Equation}

If does not have boundary conditions, ill-possed problem. The equation needs two conditions. It is very easy to partition the interval $\left[a,b\right]$.

\begin{definition}[Domain]
We call $\Omega\subset\mathbb{R}^d$ for $d=1,2,3$ a \textbf{domain} iff
\begin{enumerate}
	\item $\Omega$ is open.
	\item $\Omega$ is connected. It has no holes. It must be smooth.
\end{enumerate}
\end{definition}

\begin{definition}[Boundary]
We call $\Gamma=\partial\Omega$ the \textbf{boundary} of the domain $\Omega$.
\end{definition}

\begin{definition}[Unit normal vector]
By $\vec{n}$ we denote the \textbf{unit normal vector} (facing outwards) on the boundary.
\end{definition}

\begin{definition}
We define \textbf{function space of differentiable functions} \[ C^{m}(\Omega)=\left\{f\colon\Omega\rightarrow\mathbb{R}\mid f\left(x_1.x_2,\ldots, x_d\right)\right\}. \]
\end{definition}

\begin{definition}
We define the \textbf{Laplace operator}
\end{definition}

\begin{itemize}
	\item Let $f\in C^{0}\left(\Omega\right)$ be the \textbf{right hand side function}.
	\item Let $g\in C^{0}\left(\Gamma\right)$ be the \textbf{boundary value function}.
	\item \textbf{Dirichtlet Problem} we are looking for $u\in C^{2}(\Omega)$ such that \[ -\Delta u=f\text{ in }. \]
	\item Let $\Omega$ be the unit sphere \[ \Omega=\left\{\bm{x}=\left(x,y\right)\in\mathbb{R}^{2}\mid x^{2}+y^{2}\le 1\right\}. \]
	\item Let $f=1$ and $g=0$.
	\item There is \textbf{no solution} to the Dirichlet Problem \[-\Delta u=1\text{ in }u=0\text{ on }\Gamma \] which is $2$ times differentiable.
\end{itemize}

\section{The variational formulation}

\begin{itemize}
	\item Assume that $u\in C^{2}(\Omega)$ is a solution to the Laplace problem \[-\Delta u\left(x,y\right)=f\left(x,y\right)\text{ in }\Omega\text{ with }u=0\text{ on }\Gamma.\]
	\item Then, we can multiply this equation with a \textbf{test function} $\phi$ \[ -\Delta u\left(x,y\right)\cdot\phi\left(x,y\right)=f\left(x,y\right)\cdot\phi\left(x,y\right)\text{ in }\Omega. \]
	\item Then, we can \textbf{integrate by parts over the domain} \[ -\int_{D}\Delta u\left(x,y\right)\cdot\phi\left(x,y\right)\mathrm{d}x\mathrm{d}y=\int_{\Omega}f\left(x,y\right)\cdot\phi\left(x,y\right)\mathrm{d}x\mathrm{d}y. \]
	\item We assume that the test function is differentiable $\phi\in C^{1}(\Omega)$. Then, we can \textbf{integrate by parts} \[\int_{\Omega}\nabla u\left(x,y\right)\cdot\phi\left(x,y\right)\mathrm{d}x\mathrm{d}y-\int_{\Gamma}\left(\vec{n}\cdot\nabla\right)u\cdot\phi\mathrm{d}S=\int_{\Omega}f\left(x,y\right)\cdot\phi\left(x,y\right)\mathrm{d}x\mathrm{d}y. \]
	\item We assume that the test function is zero on the boundary. Then \[ \int_{\Omega}\nabla u\left(x,y\right)\cdot\nabla\phi\left(x,y\right)\mathrm{d}x\mathrm{d}y=\int_{\Omega}f\left(x,y\right)\cdot\phi\left(x,y\right)\mathrm{d}x\mathrm{d}y. \]
\end{itemize}

If the boundary is given by he graph of a function in $C^{2}$, then there exists a classical solution $u\in C^{2}\left(\Omega\right)$.
\begin{itemize}
	\item We introduce $L^{2}$ scalar product.
	\begin{align*}
	\left(u,\phi\right)=\int_{\Omega}.
	\end{align*}
\end{itemize}

\begin{theorem}
Let $\Omega\subset\mathbb{R}^{d}$ for $d=1,2,3$ be a domain and $f\in L^{2}\left(\Omega\right)$. Then, there exists a solution \[ u\in\mathcal{V}=H^{1}_{0}\left(\Omega\right) \] to the \textit{Laplace problem} in variational formulation.
\end{theorem}

\chapter{Finite Element Method}

\textbf{Steps for a finit element discretization}

\begin{enumerate}
	\item We discretize the domain $\Omega$ by a mesh $\Omega_{h}$.
	\item On $\Omega_{h}$ we discretize the function space $\mathcal{V}=H^{1}_{0}\left(\Omega\right)$ by a finite element space $V_{h}$.
	\item We restrict the variational formulation to $V_{h}$ \[u_{h}\in V_{h} \left(\nabla u_{h},\nabla \phi_{h}\right)=\left(f,\phi_{h}\right)\quad\forall \phi_{h}. \]
	\item We solve a linear system of equations.
\end{enumerate}

\section{Construction}

\begin{itemize}
	\item We discretize the domain $\Omega$ by splitting it into simple \textbf{open elements}, e.g, triangles, quadrilaterals (in $2D$) or tetrahedras, prisms, hexaedras, pyramids (in $3D$)
	\item The \textbf{finite element mesh} $\Omega_{h}$.
\end{itemize}

\section{Some examples}

\section{Shape assumption}

\subsection{Local Finite Element space}

\begin{itemize}
	\item On every element $T\in\Omega_{h}$ define the basis functions of a simple polynomial space.
	\item \textbf{bi-linear finite elements}.
	\item Let $T$ be and quadrilateral with the points $x^{(1)}=\left(0,0\right)$, $x^{(2)}=\left(h,0\right)$, $x^{(3)}=\left(o,h\right)$, $x^{(4)}=\left(h,h\right)$.
	\item $\phi^{(1)}\left(x,y\right)=\left(1-\frac{x}{h}\right)\left(1-\frac{y}{h}\right)$, $\phi^{(2)}\left(x,y\right)=\frac{x}{h}\left(1-\frac{y}{h}\right)$, $\phi^{(3)}\left(x,y\right)=\left(1-\frac{x}{h}\right)\frac{y}{h}$, $\phi^{(1)}\left(x,y\right)=\frac{xy}{h^{2}}$.
\end{itemize}

\begin{itemize}
\item The Lagrange basis of the finite element space is given as \[ V_{h}=\left\{\phi_{h}\in C\left(\Omega\right)\mid\restr{\phi}{T}\in Q^{1}=\operatorname{span}\left(\phi^{(1)}_{h},\phi^{(2)}_{h},\phi^{(3)}_{h},\phi^{(4)}_{h}\right)\right\}. \]
\item The \textbf{Lagrange basis} of \textbf{nodal basis} is given by \[ V. \]
\end{itemize}

\begin{itemize}
	\item Starting point: weak formulation of Laplace equation \[ u\in\mathcal{V}. \]
	\item \[ u_{h}\in V_{h}\left(\nabla u_{h},\nabla\phi_{h}\right)=\left(f,\phi_{h}\right)\quad\forall\phi_{h}\in V_{h}. \]
	\item The finite element is given by a local basis \[V_{h}=\operatorname{span}\left\{\phi^{(1)}_{h},\ldots,\phi^{(N)}_{h}\right\}\quad\forall i=1,\ldots, N. \]
	\item We write the unknown solution $u_{h}\in V_{h}$.
\end{itemize}

\section{Assembling the matrix}

\begin{itemize}
	\item We must compute the matrix entries \[ A_{ij}\left(\nabla\phi_h^{(j)}, \nabla\phi_h^{(i)}\right)=\int_{\Omega}\nabla\phi_h^{(j)}\cdot\nabla\phi_h^{(i)}\mathrm{d}x=\sum_{T\subset\Omega_h}\int_{T}\nabla\phi_h^{(j)}\cdot\nabla\phi_h^{(i)}\mathrm{d}x. \]
	\item For every \textbf{nodal}\ldots.
\end{itemize}

\begin{itemize}
	\item We combine the result in a \textbf{stencil} \[ S=\begin{bmatrix}s_{31} & s_{32} & s_{33}\\s_{21} & s_{22} & s_{23}\\s_{11} & s_{12} & s_{13}\end{bmatrix}. \]
	\item The finite element matrix on a small mesh with $16=4\cdot4$ nodes like \[ A=\frac{1}{3}\begin{bmatrix}1\end{bmatrix}. \]
\end{itemize}

\begin{itemize}
	\item The main difference between $1D$ and $2D$ (or $3D$).
\end{itemize}

\end{document}