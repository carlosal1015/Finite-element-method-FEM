% arara: pdflatex
% arara: pdflatex
\documentclass[openany,a4paper,11pt]{memoir}
\usepackage{amsmath,amssymb,bm,amsthm}

\theoremstyle{definition}
\newtheorem{definition}{Definition}
\newtheorem{theorem}{Theorem}

\newcommand\restr[2]{{% we make the whole thing an ordinary symbol
  \left.\kern-\nulldelimiterspace % automatically resize the bar with \right
  #1 % the function
  \vphantom{\big|} % pretend it's a little taller at normal size
  \right|_{#2} % this is the delimiter
  }}

\begin{document}

\chapter{Laplace Equation}

If does not have boundary conditions, ill-possed problem. The equation needs two conditions. It is very easy to partition the interval $[a,b]$.

\begin{definition}
  We call $\Omega\subset\mathbb{R}^d$ for $d=1,2,3$ a \textbf{domain} if
  \begin{enumerate}
  \item $\Omega$ is open.
  \item $\Omega$ is connected. It has no holes. It must be smooth.
  \end{enumerate}
\end{definition}

\begin{definition}
  We call $\Gamma=\partial\Omega$ the \textbf{boundary} of the domain $\Omega$.
\end{definition}

\begin{definition}
  By $\vec{n}$ we denote the \textbf{unit normal vector} (facing outwards) on the boundary.
\end{definition}

\begin{definition}
  We define \textbf{function space of differentiable functions} \[ C^m(\Omega)=\left\{ f\colon\Omega\rightarrow\mathbb{R}\mid f\left(x_1.x_2,\ldots, x_d\right) \right\}. \]
\end{definition}

\begin{definition}
  We define \textbf{laplace operator}
\end{definition}

\begin{itemize}
\item Let $f\in C^{0}(\Omega)$ be the \textbf{right hand side function}.
\item Let $g\in C^0(\Gamma)$ be the \textbf{boundary value function}.
\item \textbf{Dirichtlet Problem} we are looking for $u\in C^2(\Omega)$ such that \[ -\Delta u=f\text{ in }. \]
\item Let $\Omega$ be the unit sphere \[ \Omega=\left\{ \bm{x} = (x,y)\in\mathbb{R}^2\mid x^2+y^2\le 1 \right\}. \]
\item Let $f=1$ and $g=0$.
\item There is \textbf{no solution} to the Dirichlet Problem \[-\Delta u=1 \text{ in } u=0 \text{ on }\Gamma. \] which is $2$ times differentiable.
\end{itemize}

\section{The variational formulation}

\begin{itemize}
\item Assume that $u\in C^2(\Omega)$ is a solution to the Laplace problem \[-\Delta u(x,y)=f(x,y) \text{ in }\Omega\text{ with }u=0\text{ on }\Gamma.\]
\item Then, we can multiply this equation with a \textbf{test function} $\phi$ \[ -\Delta u(x,y)\cdot \phi(x,y)=f(x,y)\cdot\phi(x,y)\text{ in }\Omega.\] 
\item Then, we can \textbf{integrate by parts over the domain} \[-\int_{D}\Delta u(x,y)\cdot\phi(x,y)\mathrm{d}x\mathrm{d}y=\int_{\Omega}f(x,y)\cdot\phi(x,y)\mathrm{d}x\mathrm{d}y \]
\item We assume that the test function is differentiable $\phi\in C^1(\Omega)$. Then, we can \textbf{integrate by parts} \[\int_{\Omega}\nabla u(x,y)\cdot\phi(x,y)\mathrm{d}x\mathrm{d}y-\int_\Gamma(\vec{n}\cdot\nabla)u\cdot\phi\mathrm{d}S=\int_{\Omega}f(x,y)\cdot\phi(x,y)\mathrm{d}x\mathrm{d}y \]
\item We assume that the test function is zero on the boundary. Then \[\int_\Omega\nabla u(x,y)\cdot\nabla\phi(x,y)\mathrm{d}x\mathrm{d}y=\int_\Omega f(x,y)\cdot\phi(x,y)\mathrm{d}x\mathrm{d}y.\]
\end{itemize}

If the boundary is given by he graph of a function in $C^2$, then there exists a classical solution $u\in C^2(\Omega)$.
\begin{itemize}
\item We introduce $L^2$ scalar product
\begin{align*}
  (u,\phi)=\int_\Omega 
\end{align*}
\end{itemize}

\begin{theorem}
  Let $\Omega\subset\mathbb{R}^d$ for $d=1,2,3$ be a domain and $f\in L^2(\Omega)$. Then, there exists a solution \[ u\in\mathcal{V} = H^1_0(\Omega) \] to the \textit{Laplace problem} in variational formulation.
\end{theorem}

\chapter{Finite Element Method}

\textbf{Steps for a finit element discretization}

\begin{enumerate}
\item We discretize the domain $\Omega$ by a mesh $\Omega_h$.
\item On $\Omega_h$ we discretize the function space $\mathcal{V}=H^1_0(\Omega)$ by a finite element space $V_h$.
\item We restrict the variational formulation to $V_h$ \[u_h\in V_h \left(\nabla u_h,\nabla \phi_h\right)=\left(f,\phi_h\right)\quad\forall \phi_h.\]
\item We solve a linear system of equations.
\end{enumerate}

\section{Construction}

\begin{itemize}
\item We discretize the domain $\Omega$ by splitting it into simple \textbf{open elements}, e.g, triangles, quadrilaterals (in $2D$) or tetrahedras, prisms, hexaedras, pyramids (in $3D$)
\item The \textbf{finite element mesh} $\Omega_h$.
\end{itemize}

\section{Some examples}

\section{Shape assumption}

\subsection{Local Finite Element space}

\begin{itemize}
\item On every element $T\in\Omega_h$ define the basis functions of a simple polynomial space.
\item \textbf{bi-linear finite elments}
\item Let $T$ be and quadrilateral with the points $x^{(1)}=(0,0)$, $x^{(2)}=(h,0)$, $x^{(3)}=(o,h)$, $x^{(4)}=(h,h)$.
\item $\phi^{(1)}(x,y)=(1-\frac{x}{h})(1-\frac{y}{h})$, $\phi^{(2)}(x,y)=\frac{x}{h}(1-\frac{y}{h})$, $\phi^{(3)}(x,y)=(1-\frac{x}{h})\frac{y}{h}$, $\phi^{(1)}(x,y)=\frac{xy}{h^2}$
\end{itemize}

\begin{itemize}
\item The Lagrange basis of the finite element space is given as \[ V_h=\left\{\phi_h\in C\left(\Omega\right)\mid\restr{\phi}{T}\in Q^1=\mathrm{span}\left(\phi^{(1)}_h,\phi^{(2)}_h,\phi^{(3)}_h,\phi^{(4)}_h\right) \right\}\]
\item The \textbf{Lagrange basis} of \textbf{nodal basis} is given by \[ V. \]
\end{itemize}

\begin{itemize}
\item Starting point: weak formulation of Laplace equation \[u\in\mathcal{V}. \]
\item \[u_h\in V_h \left(\nabla u_h,\nabla\phi_h\right)=\left(f,\phi_h\right)\quad\forall\phi_h\in V_h. \]
\item The finite element is given by a local basis \[V_h=\text{span}\left\{\phi^{(1)}_h,\ldots,\phi^{(N)}_h\right\}\quad\forall i=1,\ldots, N. \]
\item We write the unknown solution $u_h\in V_h$.
\end{itemize}

\section{Assembling the matrix}

\begin{itemize}
\item We must compute the matrix entries \[A_{ij}\left(\nabla\phi_h^{(j)}, \nabla\phi_h^{(i)}\right)=\int_\Omega\nabla\phi_h^{(j)}\cdot\nabla\phi_h^{(i)}\mathrm{d}x=\sum_{T\subset\Omega_h}\int_T\nabla\phi_h^{(j)}\cdot\nabla\phi_h^{(i)}\mathrm{d}x. \]
\item For every \textbf{nodal} \ldots
\end{itemize}

\begin{itemize}
\item We combine the result in a \textbf{stencil} \[S=\begin{bmatrix}s_{31} & s_{32} & s_{33}\\s_{21} & s_{22} & s_{23}\\s_{11} & s_{12} & s_{13}\end{bmatrix}.\]
\item The finite element matrix on a small mesh with $16=4\cdot 4$ nodes like \[A=\frac{1}{3}\begin{bmatrix}1\end{bmatrix}.\]
\end{itemize}

\begin{itemize}
\item The main difference between $1D$ and $2D$ (or $3D$).
\end{itemize}
\end{document}