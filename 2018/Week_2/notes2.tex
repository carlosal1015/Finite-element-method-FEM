% arara: clean: {
% arara: --> extensions:
% arara: --> ['log','aux','synctex.gz','out']
% arara: --> }
% arara: lualatex: { draft: yes }
% arara: lualatex: {
% arara: --> shell: yes,
% arara: --> synctex: yes,
% arara: --> interaction: batchmode
% arara: --> }
% arara: clean: {
% arara: --> extensions:
% arara: --> ['log','aux','synctex.gz','out']
% arara: --> }
\documentclass[
	a4paper,
	11pt,
	oneside
]{scrreprt}

\usepackage{mathpazo}
\usepackage{fontspec}
\setmainfont{TeX Gyre Pagella}
\usepackage{amsmath,amssymb,amsthm,bm}

\theoremstyle{definition}
\newtheorem{definition}{Definition}
\newtheorem{theorem}{Theorem}

\newcommand\restr[2]{{% we make the whole thing an ordinary symbol
		\left.\kern-\nulldelimiterspace % automatically resize the bar with \right
		#1 % the function
		\vphantom{\big|} % pretend it's a little taller at normal size
		\right|_{#2} % this is the delimiter
}}

\usepackage[
	colorlinks,
	citecolor=blue,
	linkcolor=red,
	pdfpagelabels
]{hyperref}

\hypersetup{pageanchor=false}

\begin{document}

\chapter{$P_1$ Finite elements on triangles - tetrahedrons}

Let $V=H^{0}_{0}\left(\Omega\right)$. We consider the Poisson problem $u\in H^{1}_{0}\left(\Omega\right)$ \[-\Delta u=f\text{ in }\Omega \] and its weak formulation %TODO: Convection field

\section{Lecture $2$: Accuracy of Finite Element Discretization}

\textbf{Outline for today:}

\begin{enumerate}
	\item A priori error estimates in $H^{1}$ \[ {\|\nabla(u-u_{h})\|}_{L^{2}\left(h\right)}. \]
\end{enumerate}

$C$ dependes of the solution. One is usually interest in the error \[ {\|u-u_{h}\|}_{X}=? \] for a certain norm ${\|\cdot\|}_{X}$.

\begin{description}
	\item[A priori error estimates:] Information about the error in terms of mesh size asymptotics, e.g, for $P_{1}$ or $Q_{1}$ elements
	\begin{align*}
	{ \|\nabla\left(u-u_{h}\right)\|}_{L^{2}(h)}	&	\le ch{|u|}_{H^{2}(\Omega)} \\
	\|u-u_{h}\|_{L^{2}(\Omega)}										&	\le ch^{2}{|u|}_{H^2(\Omega)}.
	\end{align*}
	\item[A posteriori error estimates:] Information about the error in terms of $u_{h}$, e.g, \[ \|\nabla(u-u_{h})\|^{2}_{L^{2}(\Omega)}\le \sum_{}. \]
\end{description}

\section{Galerkin Orthogonality}

\begin{itemize}
	\item Continuous problem with $A\colon V\times V\rightarrow\mathbb{R}$ bilinear: \[ u\in V\colon A\left(u,\phi\right)=\left(f,\phi\right)\quad\forall\phi\in V. \] Most simple example: \[ A\left(u,\phi\right)=\left(\nabla u,\nabla\phi\right)=\int_{\Omega}\nabla u\nabla\phi\mathrm{d}x. \]
	\item Discrete problem: \[ u_{h}\in V_{N}\colon A\left(u_h,\phi\right)=\left(f,\phi\right)\quad\forall\phi\in V_{h}. \]
	\item Discretization error \[ e_{h}=. \]
\end{itemize}

\begin{theorem}[Cea's lemma]
Suppose that the bilinear form $A\colon V\times V\rightarrow\mathbb{R}$ satisfies the conditions of \linebreak Lax--Milgram's theorem (continuous and $V$-coercive with $\alpha_{1},\alpha_{2}>0$). Further, let $V_{h}\subseteq V$ a subspace. Then: \[ \|u-u_{h}\|_{V}\le \frac{\alpha_{1}}{\alpha_{2}}\inf_{v_h\in V}. \]
\end{theorem}

\section{Continuity and coercivity}

\begin{description}
	\item[Continuity:] There exists $\alpha_{1}\ge0$ such that \[ A\left(u,\phi\right)\le\alpha_{1}. \]
\end{description}

\begin{proof}
Let $v_{h}\in V_{h}$ be arbitrary,
\begin{align*}
	\alpha_{2}{\|u-u_{h}\|}^{2}_{V}	&\le A\left(u-u_{h}, u-u_{h}\right)\quad (\text{ coercivity })\\
																	&
\end{align*}
\end{proof}

\begin{itemize}
	\item $V=H^{1}_{0}(\Omega)$.
	\item associated norm \[ \|u\|_{V}={\|\nabla u\|}_{\Omega}={\left(\int_{\Omega}{|\nabla u(x)|}^{2}\mathrm{d}x\right)}^{1/2}. \]
	\item $\alpha_{1}=\alpha_{2}=1$ \[ {\|\nabla\left(u-u_{h}\right)\|}_{\Omega}=\inf_{v_{h}\in V_{h}}{\|\nabla\left(u-v_{h}\right)\|}_{\Omega}. \]
\end{itemize}

\section{Interpolation error}

\begin{itemize}
	\item Let $I_{h}\colon V\rightarrow V_{h}$ be an arbitrary interpolation. Then
	\begin{align*}
	\|u-u_h\|_{V}	&\le\frac{\alpha_1}{\alpha_2}\inf_{v_h\in V_h}{\|u-v_h\|}_{V}\\
								&\le \frac{\alpha_1}{\alpha_2}{\|u-I_hu\|}_{V}.
	\end{align*}
	\item We only need do get and idea about the interpolation error \[ {\|u-I_{h}u\|}_{V}. \]
	\item Most simple is the nodal interpolation of continuous functions \[ I_{h}u(N)=u(N) \] for nodes $N$ of the mesh.
	\item But: Are $H^{1}(\Omega)$ functions continuous? \[ d=1\quad\left(\text{yes}\right)\qquad d\ge2\quad\left(\text{no}\right). \]
\end{itemize}

\section{More regular Sobolev functions}

\begin{itemize}
	\item Higher order Sobolev spaces of order $k\ge 1$: \[ H^{k}(\Omega)=. \]
\end{itemize}

\section{$H^{2}$ functions are continuous}

\begin{itemize}
	\item For $d=1$ \[ H^{1}(\Omega)\subseteq C(\Omega). \]
	\item For $d=2$ and $d=3$ \[ H^2(\Omega)\subseteq C(\Omega). \]
	\item If $\partial\Omega$ is Lipschitz \ldots.
\end{itemize}

\section{Nodal interpolation of $H^{2}$-functions}

Hence, if $u\in H^{2}(\Omega)$, then it holds for the Poisson pb \[ \|\nabla (u-I_{h})\|. \]

\section{Structure to address the interpolation error}

\begin{enumerate}
	\item Location: \[ {\|\nabla(u-I_{h}u)\|}^{2}_{\Omega}=\sum_{T\in T_{h}}{\|\nabla(u-I_{h}u)\|}^{2}_{T}. \]
	\item Transformation to the reference cell: \[ \|\nabla\left(u-I_{h}\right)\|. \]
\end{enumerate}

\section{Step $2$: Transformation to the reference cell}

How to transform an expression as $\left(w=u-I_{h}u\right)$ \[ {\|\nabla\left(u-I_{h}u\right)\|}^{2}_{T}=\int_{T}{\left|\nabla w(x)\right|}^{2}\mathrm{d}x \] onto the reference triangle $\hat{T}$ by an affine linear transformation \[ \phi_{T}\left(\hat{x}\right)=x_{0}+B_{T}\hat{x}. \]

Partial derivative

\begin{align*}
\frac{\partial w\left(x\right)}{\partial x_{i}}=\sum_{j=1}^{d}
\end{align*}

Gradient oin $T$: \[ {\left|\nabla w(x)\right|}^2\le{\|B_{T}^{-t}\|}^{2}_{F}{\left|\hat{\nabla}\hat{w}(\hat{x})\right|}^{2} \] with Frobenius norm $\|B_{T}^{-t}\|$.

\section{Step $3$: Interpolation error on the reference cell}

\begin{theorem}[Bramble-Hilbert lemma]
Let $T\subset\mathbb{R}^{d}$ a Lipschitz domain, $F$ a normed space, $\phi\colon H^{m}\left(T\right)\rightarrow F$.
\end{theorem}

\section{Step $4$: Backward transformation}

\begin{align*}
{\left|\hat{u}\right|}_{H^{2}\left(\hat{T}\right)}&={\left|\operatorname{det}B_{T}\right|}^{-1/2}\|B_{T}^{t}\|.
\end{align*}

\section{Geometrical parameters}

Let $h_{T}=$ an outer radius, $\rho_{T}$ an inner radius and $\kappa_{T}=\frac{h_{T}}{\rho_{T}} $ the aspect ratio.

\begin{itemize}
	\item A family of meshes $\mathcal{T}_{1}$, $\mathcal{T}_{2}$, $\mathcal{T}_{3},\ldots$ is called \textbf{shape regular} if $\max_{i}\max_{T\in\mathcal{T}_{i}}\kappa_{T}\le\kappa$.
	\item A family of meshes $\mathcal{T}_{1}$, $\mathcal{T}_{2}$, $\mathcal{T}_{3},\ldots$ is called \textbf{quasi uniform} if $\frac{\max}{\min}\le\kappa$.
\end{itemize}

\section{Spectral norm of transformation}

For the spectral norm of the affin linear transformation $\phi_{T}$

\section{Step $5$: Assembling together}

\[ {\|\nabla\left(u-I_{h}u\right)\|}^{2}_{\Omega}\le c\sum_{T\in\mathcal{T}_{h}}{\|B_{T}^{-1}\|}^{2}_{F}\|B_{T}\|. \]

Error estimates in the $L^{2}$ norm

\section{$L^{2}$ error estimates}

Are you interested in a ``weaker'' norm \[ {\|u-u_{h}\|}_{L^{2}\left(\Omega\right)} \] instead of \[ {\|u-I_{h}u\|}_{L^{2}\left(\Omega\right)}\le c_{\kappa}\bm{h^{2}}|u|H^{2}\left(\Omega\right) \] ?

%TODO: Desigualdad de Poincaré.

\section{Duality argument}

\begin{description}
	\item[Aim:] Derive error bound on \[ {\|u-u_{h}\|}_{W} \] in a weaker norm, i.e. let $W$ be an Hilbert space with continuous embedding $V\subseteq W$, i.e. \[ {\|u\|}_{W}. \]
\end{description}

\begin{itemize}
	\item Due to the continuous embedding $V\subseteq W$ it holds \[ W^{\prime}\subseteq V^{\prime}. \]
	\item Hence, $g\in S\subset V^{\prime}$ is a possible rhs in the dual problem: \[ z_{g}\in V\colon\quad A\left(\phi, z_{g}\right)=\langle g,\phi\rangle\quad\forall\phi\in V. \]
	\item Primal problem: \[ u\in V\colon\quad A\left(u,\phi\right)=\langle f,\phi\rangle. \]
\end{itemize}

%http://math.oregonstate.edu/~mpesz/teaching/659_S04/notes_AN.pdf
\section{Aubin-Nitsche trick}
We arribe at:
\begin{theorem}
Let $\Omega\subset\mathbb{R}^{d},d\in\left\{2,3\right\}$, be a convex domain or a domain with $C^{2}$--boundary, $\{T_{h}\}$.
\end{theorem}

\section{Higher order finite elements}

\begin{itemize}
	\item FEM or order $r\ge1$: \[ P_{r}\left(\mathcal{T}_{h}\right)=\left\{\varphi\colon\Omega_{h}\rightarrow\mathbb{R}\colon\restr{\varphi}{T}\in P_r\forall T\in\mathcal{T}_h \right\}. \]
\end{itemize}
%TODO: No necesariamente coercivo.

\section{Error estimate for higher order finite elements}

\begin{theorem}
We consider the Poisson problem, discretized with $P_{r}$ finite elements $\left(r\ge1\right)$ on a family of shape regular meshes. If the solution $u$ has regularity $H^{r+1}$, then \[ {\|\nabla\left(u-I_{h}u\right)\|}_{\Omega}\le c_{\kappa}h^{r}\left|u\right|H^{r+1}. \] %\ldots.
\end{theorem}

\section{Pro's and cons of higher order finite elements}

\textbf{Pro's}:
\begin{itemize}
	\item A better approximation property is expected due to better asymptotic behavior.
	\item Less degrees of freedom for a given accuracy.
	\item More local couplings in the stiffness matrix (can be advantageous for CPU reasons).
\end{itemize}

\textbf{Contra}:
\begin{itemize}
	\item More regularity of the solution is necessary. Otherwise: reduction of accuracy/order of convergence.
	\item Stiffness matrix become more dense due to many couplings inside each elements.
	\item Robust linear solvers are usually more difficult.
\end{itemize}

\section{Accuracy of $Q_{r}$ elements}

\[ \varphi\left(x,y\right)=\sum_{i,j=0}^{r}\alpha_{ij}x^{i}y^{j}. \]

\begin{itemize}
	\item The nodal interpolation $\hat{I}$ in the reference quadrilateral \textbackslash hexahedral is exacts for polynomials of degree $\le r$.
\end{itemize}

\section{Summary of Lecture $2$}

\begin{itemize}
	\item FE for continuous, coercive bilinear forms are quasi-optimal with respect to discretization error:
\end{itemize}

Abcedario.

\end{document}