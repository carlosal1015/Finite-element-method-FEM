
% Default to the notebook output style

    


% Inherit from the specified cell style.




    
\documentclass[11pt]{article}

    
    
    \usepackage[T1]{fontenc}
    % Nicer default font (+ math font) than Computer Modern for most use cases
    \usepackage{mathpazo}

    % Basic figure setup, for now with no caption control since it's done
    % automatically by Pandoc (which extracts ![](path) syntax from Markdown).
    \usepackage{graphicx}
    % We will generate all images so they have a width \maxwidth. This means
    % that they will get their normal width if they fit onto the page, but
    % are scaled down if they would overflow the margins.
    \makeatletter
    \def\maxwidth{\ifdim\Gin@nat@width>\linewidth\linewidth
    \else\Gin@nat@width\fi}
    \makeatother
    \let\Oldincludegraphics\includegraphics
    % Set max figure width to be 80% of text width, for now hardcoded.
    \renewcommand{\includegraphics}[1]{\Oldincludegraphics[width=.8\maxwidth]{#1}}
    % Ensure that by default, figures have no caption (until we provide a
    % proper Figure object with a Caption API and a way to capture that
    % in the conversion process - todo).
    \usepackage{caption}
    \DeclareCaptionLabelFormat{nolabel}{}
    \captionsetup{labelformat=nolabel}

    \usepackage{adjustbox} % Used to constrain images to a maximum size 
    \usepackage{xcolor} % Allow colors to be defined
    \usepackage{enumerate} % Needed for markdown enumerations to work
    \usepackage{geometry} % Used to adjust the document margins
    \usepackage{amsmath} % Equations
    \usepackage{amssymb} % Equations
    \usepackage{textcomp} % defines textquotesingle
    % Hack from http://tex.stackexchange.com/a/47451/13684:
    \AtBeginDocument{%
        \def\PYZsq{\textquotesingle}% Upright quotes in Pygmentized code
    }
    \usepackage{upquote} % Upright quotes for verbatim code
    \usepackage{eurosym} % defines \euro
    \usepackage[mathletters]{ucs} % Extended unicode (utf-8) support
    \usepackage[utf8x]{inputenc} % Allow utf-8 characters in the tex document
    \usepackage{fancyvrb} % verbatim replacement that allows latex
    \usepackage{grffile} % extends the file name processing of package graphics 
                         % to support a larger range 
    % The hyperref package gives us a pdf with properly built
    % internal navigation ('pdf bookmarks' for the table of contents,
    % internal cross-reference links, web links for URLs, etc.)
    \usepackage{hyperref}
    \usepackage{longtable} % longtable support required by pandoc >1.10
    \usepackage{booktabs}  % table support for pandoc > 1.12.2
    \usepackage[inline]{enumitem} % IRkernel/repr support (it uses the enumerate* environment)
    \usepackage[normalem]{ulem} % ulem is needed to support strikethroughs (\sout)
                                % normalem makes italics be italics, not underlines
    

    
    
    % Colors for the hyperref package
    \definecolor{urlcolor}{rgb}{0,.145,.698}
    \definecolor{linkcolor}{rgb}{.71,0.21,0.01}
    \definecolor{citecolor}{rgb}{.12,.54,.11}

    % ANSI colors
    \definecolor{ansi-black}{HTML}{3E424D}
    \definecolor{ansi-black-intense}{HTML}{282C36}
    \definecolor{ansi-red}{HTML}{E75C58}
    \definecolor{ansi-red-intense}{HTML}{B22B31}
    \definecolor{ansi-green}{HTML}{00A250}
    \definecolor{ansi-green-intense}{HTML}{007427}
    \definecolor{ansi-yellow}{HTML}{DDB62B}
    \definecolor{ansi-yellow-intense}{HTML}{B27D12}
    \definecolor{ansi-blue}{HTML}{208FFB}
    \definecolor{ansi-blue-intense}{HTML}{0065CA}
    \definecolor{ansi-magenta}{HTML}{D160C4}
    \definecolor{ansi-magenta-intense}{HTML}{A03196}
    \definecolor{ansi-cyan}{HTML}{60C6C8}
    \definecolor{ansi-cyan-intense}{HTML}{258F8F}
    \definecolor{ansi-white}{HTML}{C5C1B4}
    \definecolor{ansi-white-intense}{HTML}{A1A6B2}

    % commands and environments needed by pandoc snippets
    % extracted from the output of `pandoc -s`
    \providecommand{\tightlist}{%
      \setlength{\itemsep}{0pt}\setlength{\parskip}{0pt}}
    \DefineVerbatimEnvironment{Highlighting}{Verbatim}{commandchars=\\\{\}}
    % Add ',fontsize=\small' for more characters per line
    \newenvironment{Shaded}{}{}
    \newcommand{\KeywordTok}[1]{\textcolor[rgb]{0.00,0.44,0.13}{\textbf{{#1}}}}
    \newcommand{\DataTypeTok}[1]{\textcolor[rgb]{0.56,0.13,0.00}{{#1}}}
    \newcommand{\DecValTok}[1]{\textcolor[rgb]{0.25,0.63,0.44}{{#1}}}
    \newcommand{\BaseNTok}[1]{\textcolor[rgb]{0.25,0.63,0.44}{{#1}}}
    \newcommand{\FloatTok}[1]{\textcolor[rgb]{0.25,0.63,0.44}{{#1}}}
    \newcommand{\CharTok}[1]{\textcolor[rgb]{0.25,0.44,0.63}{{#1}}}
    \newcommand{\StringTok}[1]{\textcolor[rgb]{0.25,0.44,0.63}{{#1}}}
    \newcommand{\CommentTok}[1]{\textcolor[rgb]{0.38,0.63,0.69}{\textit{{#1}}}}
    \newcommand{\OtherTok}[1]{\textcolor[rgb]{0.00,0.44,0.13}{{#1}}}
    \newcommand{\AlertTok}[1]{\textcolor[rgb]{1.00,0.00,0.00}{\textbf{{#1}}}}
    \newcommand{\FunctionTok}[1]{\textcolor[rgb]{0.02,0.16,0.49}{{#1}}}
    \newcommand{\RegionMarkerTok}[1]{{#1}}
    \newcommand{\ErrorTok}[1]{\textcolor[rgb]{1.00,0.00,0.00}{\textbf{{#1}}}}
    \newcommand{\NormalTok}[1]{{#1}}
    
    % Additional commands for more recent versions of Pandoc
    \newcommand{\ConstantTok}[1]{\textcolor[rgb]{0.53,0.00,0.00}{{#1}}}
    \newcommand{\SpecialCharTok}[1]{\textcolor[rgb]{0.25,0.44,0.63}{{#1}}}
    \newcommand{\VerbatimStringTok}[1]{\textcolor[rgb]{0.25,0.44,0.63}{{#1}}}
    \newcommand{\SpecialStringTok}[1]{\textcolor[rgb]{0.73,0.40,0.53}{{#1}}}
    \newcommand{\ImportTok}[1]{{#1}}
    \newcommand{\DocumentationTok}[1]{\textcolor[rgb]{0.73,0.13,0.13}{\textit{{#1}}}}
    \newcommand{\AnnotationTok}[1]{\textcolor[rgb]{0.38,0.63,0.69}{\textbf{\textit{{#1}}}}}
    \newcommand{\CommentVarTok}[1]{\textcolor[rgb]{0.38,0.63,0.69}{\textbf{\textit{{#1}}}}}
    \newcommand{\VariableTok}[1]{\textcolor[rgb]{0.10,0.09,0.49}{{#1}}}
    \newcommand{\ControlFlowTok}[1]{\textcolor[rgb]{0.00,0.44,0.13}{\textbf{{#1}}}}
    \newcommand{\OperatorTok}[1]{\textcolor[rgb]{0.40,0.40,0.40}{{#1}}}
    \newcommand{\BuiltInTok}[1]{{#1}}
    \newcommand{\ExtensionTok}[1]{{#1}}
    \newcommand{\PreprocessorTok}[1]{\textcolor[rgb]{0.74,0.48,0.00}{{#1}}}
    \newcommand{\AttributeTok}[1]{\textcolor[rgb]{0.49,0.56,0.16}{{#1}}}
    \newcommand{\InformationTok}[1]{\textcolor[rgb]{0.38,0.63,0.69}{\textbf{\textit{{#1}}}}}
    \newcommand{\WarningTok}[1]{\textcolor[rgb]{0.38,0.63,0.69}{\textbf{\textit{{#1}}}}}
    
    
    % Define a nice break command that doesn't care if a line doesn't already
    % exist.
    \def\br{\hspace*{\fill} \\* }
    % Math Jax compatability definitions
    \def\gt{>}
    \def\lt{<}
    % Document parameters
    \title{Functions\_in\_MATLAB\_II}
    
    
    

    % Pygments definitions
    
\makeatletter
\def\PY@reset{\let\PY@it=\relax \let\PY@bf=\relax%
    \let\PY@ul=\relax \let\PY@tc=\relax%
    \let\PY@bc=\relax \let\PY@ff=\relax}
\def\PY@tok#1{\csname PY@tok@#1\endcsname}
\def\PY@toks#1+{\ifx\relax#1\empty\else%
    \PY@tok{#1}\expandafter\PY@toks\fi}
\def\PY@do#1{\PY@bc{\PY@tc{\PY@ul{%
    \PY@it{\PY@bf{\PY@ff{#1}}}}}}}
\def\PY#1#2{\PY@reset\PY@toks#1+\relax+\PY@do{#2}}

\expandafter\def\csname PY@tok@w\endcsname{\def\PY@tc##1{\textcolor[rgb]{0.73,0.73,0.73}{##1}}}
\expandafter\def\csname PY@tok@c\endcsname{\let\PY@it=\textit\def\PY@tc##1{\textcolor[rgb]{0.25,0.50,0.50}{##1}}}
\expandafter\def\csname PY@tok@cp\endcsname{\def\PY@tc##1{\textcolor[rgb]{0.74,0.48,0.00}{##1}}}
\expandafter\def\csname PY@tok@k\endcsname{\let\PY@bf=\textbf\def\PY@tc##1{\textcolor[rgb]{0.00,0.50,0.00}{##1}}}
\expandafter\def\csname PY@tok@kp\endcsname{\def\PY@tc##1{\textcolor[rgb]{0.00,0.50,0.00}{##1}}}
\expandafter\def\csname PY@tok@kt\endcsname{\def\PY@tc##1{\textcolor[rgb]{0.69,0.00,0.25}{##1}}}
\expandafter\def\csname PY@tok@o\endcsname{\def\PY@tc##1{\textcolor[rgb]{0.40,0.40,0.40}{##1}}}
\expandafter\def\csname PY@tok@ow\endcsname{\let\PY@bf=\textbf\def\PY@tc##1{\textcolor[rgb]{0.67,0.13,1.00}{##1}}}
\expandafter\def\csname PY@tok@nb\endcsname{\def\PY@tc##1{\textcolor[rgb]{0.00,0.50,0.00}{##1}}}
\expandafter\def\csname PY@tok@nf\endcsname{\def\PY@tc##1{\textcolor[rgb]{0.00,0.00,1.00}{##1}}}
\expandafter\def\csname PY@tok@nc\endcsname{\let\PY@bf=\textbf\def\PY@tc##1{\textcolor[rgb]{0.00,0.00,1.00}{##1}}}
\expandafter\def\csname PY@tok@nn\endcsname{\let\PY@bf=\textbf\def\PY@tc##1{\textcolor[rgb]{0.00,0.00,1.00}{##1}}}
\expandafter\def\csname PY@tok@ne\endcsname{\let\PY@bf=\textbf\def\PY@tc##1{\textcolor[rgb]{0.82,0.25,0.23}{##1}}}
\expandafter\def\csname PY@tok@nv\endcsname{\def\PY@tc##1{\textcolor[rgb]{0.10,0.09,0.49}{##1}}}
\expandafter\def\csname PY@tok@no\endcsname{\def\PY@tc##1{\textcolor[rgb]{0.53,0.00,0.00}{##1}}}
\expandafter\def\csname PY@tok@nl\endcsname{\def\PY@tc##1{\textcolor[rgb]{0.63,0.63,0.00}{##1}}}
\expandafter\def\csname PY@tok@ni\endcsname{\let\PY@bf=\textbf\def\PY@tc##1{\textcolor[rgb]{0.60,0.60,0.60}{##1}}}
\expandafter\def\csname PY@tok@na\endcsname{\def\PY@tc##1{\textcolor[rgb]{0.49,0.56,0.16}{##1}}}
\expandafter\def\csname PY@tok@nt\endcsname{\let\PY@bf=\textbf\def\PY@tc##1{\textcolor[rgb]{0.00,0.50,0.00}{##1}}}
\expandafter\def\csname PY@tok@nd\endcsname{\def\PY@tc##1{\textcolor[rgb]{0.67,0.13,1.00}{##1}}}
\expandafter\def\csname PY@tok@s\endcsname{\def\PY@tc##1{\textcolor[rgb]{0.73,0.13,0.13}{##1}}}
\expandafter\def\csname PY@tok@sd\endcsname{\let\PY@it=\textit\def\PY@tc##1{\textcolor[rgb]{0.73,0.13,0.13}{##1}}}
\expandafter\def\csname PY@tok@si\endcsname{\let\PY@bf=\textbf\def\PY@tc##1{\textcolor[rgb]{0.73,0.40,0.53}{##1}}}
\expandafter\def\csname PY@tok@se\endcsname{\let\PY@bf=\textbf\def\PY@tc##1{\textcolor[rgb]{0.73,0.40,0.13}{##1}}}
\expandafter\def\csname PY@tok@sr\endcsname{\def\PY@tc##1{\textcolor[rgb]{0.73,0.40,0.53}{##1}}}
\expandafter\def\csname PY@tok@ss\endcsname{\def\PY@tc##1{\textcolor[rgb]{0.10,0.09,0.49}{##1}}}
\expandafter\def\csname PY@tok@sx\endcsname{\def\PY@tc##1{\textcolor[rgb]{0.00,0.50,0.00}{##1}}}
\expandafter\def\csname PY@tok@m\endcsname{\def\PY@tc##1{\textcolor[rgb]{0.40,0.40,0.40}{##1}}}
\expandafter\def\csname PY@tok@gh\endcsname{\let\PY@bf=\textbf\def\PY@tc##1{\textcolor[rgb]{0.00,0.00,0.50}{##1}}}
\expandafter\def\csname PY@tok@gu\endcsname{\let\PY@bf=\textbf\def\PY@tc##1{\textcolor[rgb]{0.50,0.00,0.50}{##1}}}
\expandafter\def\csname PY@tok@gd\endcsname{\def\PY@tc##1{\textcolor[rgb]{0.63,0.00,0.00}{##1}}}
\expandafter\def\csname PY@tok@gi\endcsname{\def\PY@tc##1{\textcolor[rgb]{0.00,0.63,0.00}{##1}}}
\expandafter\def\csname PY@tok@gr\endcsname{\def\PY@tc##1{\textcolor[rgb]{1.00,0.00,0.00}{##1}}}
\expandafter\def\csname PY@tok@ge\endcsname{\let\PY@it=\textit}
\expandafter\def\csname PY@tok@gs\endcsname{\let\PY@bf=\textbf}
\expandafter\def\csname PY@tok@gp\endcsname{\let\PY@bf=\textbf\def\PY@tc##1{\textcolor[rgb]{0.00,0.00,0.50}{##1}}}
\expandafter\def\csname PY@tok@go\endcsname{\def\PY@tc##1{\textcolor[rgb]{0.53,0.53,0.53}{##1}}}
\expandafter\def\csname PY@tok@gt\endcsname{\def\PY@tc##1{\textcolor[rgb]{0.00,0.27,0.87}{##1}}}
\expandafter\def\csname PY@tok@err\endcsname{\def\PY@bc##1{\setlength{\fboxsep}{0pt}\fcolorbox[rgb]{1.00,0.00,0.00}{1,1,1}{\strut ##1}}}
\expandafter\def\csname PY@tok@kc\endcsname{\let\PY@bf=\textbf\def\PY@tc##1{\textcolor[rgb]{0.00,0.50,0.00}{##1}}}
\expandafter\def\csname PY@tok@kd\endcsname{\let\PY@bf=\textbf\def\PY@tc##1{\textcolor[rgb]{0.00,0.50,0.00}{##1}}}
\expandafter\def\csname PY@tok@kn\endcsname{\let\PY@bf=\textbf\def\PY@tc##1{\textcolor[rgb]{0.00,0.50,0.00}{##1}}}
\expandafter\def\csname PY@tok@kr\endcsname{\let\PY@bf=\textbf\def\PY@tc##1{\textcolor[rgb]{0.00,0.50,0.00}{##1}}}
\expandafter\def\csname PY@tok@bp\endcsname{\def\PY@tc##1{\textcolor[rgb]{0.00,0.50,0.00}{##1}}}
\expandafter\def\csname PY@tok@fm\endcsname{\def\PY@tc##1{\textcolor[rgb]{0.00,0.00,1.00}{##1}}}
\expandafter\def\csname PY@tok@vc\endcsname{\def\PY@tc##1{\textcolor[rgb]{0.10,0.09,0.49}{##1}}}
\expandafter\def\csname PY@tok@vg\endcsname{\def\PY@tc##1{\textcolor[rgb]{0.10,0.09,0.49}{##1}}}
\expandafter\def\csname PY@tok@vi\endcsname{\def\PY@tc##1{\textcolor[rgb]{0.10,0.09,0.49}{##1}}}
\expandafter\def\csname PY@tok@vm\endcsname{\def\PY@tc##1{\textcolor[rgb]{0.10,0.09,0.49}{##1}}}
\expandafter\def\csname PY@tok@sa\endcsname{\def\PY@tc##1{\textcolor[rgb]{0.73,0.13,0.13}{##1}}}
\expandafter\def\csname PY@tok@sb\endcsname{\def\PY@tc##1{\textcolor[rgb]{0.73,0.13,0.13}{##1}}}
\expandafter\def\csname PY@tok@sc\endcsname{\def\PY@tc##1{\textcolor[rgb]{0.73,0.13,0.13}{##1}}}
\expandafter\def\csname PY@tok@dl\endcsname{\def\PY@tc##1{\textcolor[rgb]{0.73,0.13,0.13}{##1}}}
\expandafter\def\csname PY@tok@s2\endcsname{\def\PY@tc##1{\textcolor[rgb]{0.73,0.13,0.13}{##1}}}
\expandafter\def\csname PY@tok@sh\endcsname{\def\PY@tc##1{\textcolor[rgb]{0.73,0.13,0.13}{##1}}}
\expandafter\def\csname PY@tok@s1\endcsname{\def\PY@tc##1{\textcolor[rgb]{0.73,0.13,0.13}{##1}}}
\expandafter\def\csname PY@tok@mb\endcsname{\def\PY@tc##1{\textcolor[rgb]{0.40,0.40,0.40}{##1}}}
\expandafter\def\csname PY@tok@mf\endcsname{\def\PY@tc##1{\textcolor[rgb]{0.40,0.40,0.40}{##1}}}
\expandafter\def\csname PY@tok@mh\endcsname{\def\PY@tc##1{\textcolor[rgb]{0.40,0.40,0.40}{##1}}}
\expandafter\def\csname PY@tok@mi\endcsname{\def\PY@tc##1{\textcolor[rgb]{0.40,0.40,0.40}{##1}}}
\expandafter\def\csname PY@tok@il\endcsname{\def\PY@tc##1{\textcolor[rgb]{0.40,0.40,0.40}{##1}}}
\expandafter\def\csname PY@tok@mo\endcsname{\def\PY@tc##1{\textcolor[rgb]{0.40,0.40,0.40}{##1}}}
\expandafter\def\csname PY@tok@ch\endcsname{\let\PY@it=\textit\def\PY@tc##1{\textcolor[rgb]{0.25,0.50,0.50}{##1}}}
\expandafter\def\csname PY@tok@cm\endcsname{\let\PY@it=\textit\def\PY@tc##1{\textcolor[rgb]{0.25,0.50,0.50}{##1}}}
\expandafter\def\csname PY@tok@cpf\endcsname{\let\PY@it=\textit\def\PY@tc##1{\textcolor[rgb]{0.25,0.50,0.50}{##1}}}
\expandafter\def\csname PY@tok@c1\endcsname{\let\PY@it=\textit\def\PY@tc##1{\textcolor[rgb]{0.25,0.50,0.50}{##1}}}
\expandafter\def\csname PY@tok@cs\endcsname{\let\PY@it=\textit\def\PY@tc##1{\textcolor[rgb]{0.25,0.50,0.50}{##1}}}

\def\PYZbs{\char`\\}
\def\PYZus{\char`\_}
\def\PYZob{\char`\{}
\def\PYZcb{\char`\}}
\def\PYZca{\char`\^}
\def\PYZam{\char`\&}
\def\PYZlt{\char`\<}
\def\PYZgt{\char`\>}
\def\PYZsh{\char`\#}
\def\PYZpc{\char`\%}
\def\PYZdl{\char`\$}
\def\PYZhy{\char`\-}
\def\PYZsq{\char`\'}
\def\PYZdq{\char`\"}
\def\PYZti{\char`\~}
% for compatibility with earlier versions
\def\PYZat{@}
\def\PYZlb{[}
\def\PYZrb{]}
\makeatother


    % Exact colors from NB
    \definecolor{incolor}{rgb}{0.0, 0.0, 0.5}
    \definecolor{outcolor}{rgb}{0.545, 0.0, 0.0}



    
    % Prevent overflowing lines due to hard-to-break entities
    \sloppy 
    % Setup hyperref package
    \hypersetup{
      breaklinks=true,  % so long urls are correctly broken across lines
      colorlinks=true,
      urlcolor=urlcolor,
      linkcolor=linkcolor,
      citecolor=citecolor,
      }
    % Slightly bigger margins than the latex defaults
    
    \geometry{verbose,tmargin=1in,bmargin=1in,lmargin=1in,rmargin=1in}
    
    

    \begin{document}
    
    
    \maketitle
    
    

    
    \hypertarget{teoruxeda-de-elementos-finitos-y-su-implementaciuxf3n-11122018}{%
\section{Teoría de elementos finitos y su implementación
(11/12/2018)}\label{teoruxeda-de-elementos-finitos-y-su-implementaciuxf3n-11122018}}

\hypertarget{funciones-que-realizan-tareas}{%
\subsection{Funciones que realizan
tareas}\label{funciones-que-realizan-tareas}}

\hypertarget{programas-usando-archivos-m-file}{%
\subsubsection{\texorpdfstring{Programas usando archivos
\(M\)-file}{Programas usando archivos M-file}}\label{programas-usando-archivos-m-file}}

\begin{itemize}
\tightlist
\item
  Archivos de instrucciones
\item
  Archivos de funciones
\end{itemize}

    \hypertarget{archivos-de-instrucciones}{%
\subsubsection{Archivos de
instrucciones}\label{archivos-de-instrucciones}}

Un archivo o \emph{script} de instrucciones consisten de una sucesión de
instrucciones normales de MATLAB.

Las variables en un archivo de instrucciones son globales y, por tanto,
cambiarán los valores del espacio de trabajo.

Los archivos de instrucciones son utilizados a menudo para introducir
datos en una matriz grande.

    \hypertarget{archivos-de-funciones}{%
\subsubsection{Archivos de funciones}\label{archivos-de-funciones}}

Los archivos de funciones hacen que MATLAB tenga \emph{capacidad de
crecimiento}.

Se pueden crear funciones específicas para un problema concreto y a
partir de su introducción, tendrán el mismo rango que las demás
funciones del sistema.

Las variables en las funciones son locales.

También se puede declararse una variable como global.

    \begin{Verbatim}[commandchars=\\\{\}]
{\color{incolor}In [{\color{incolor}1}]:} \PY{c}{\PYZpc{} Nombre del script figura1.m}
        \PY{c}{\PYZpc{} Ejecutar en el prompt de MATLAB como figura1.m}
        \PY{c}{\PYZpc{} figura 1}
        \PY{c}{\PYZpc{} shg}
\end{Verbatim}


    \hypertarget{para-direccionar-una-carpeta-de-trabajo}{%
\subsection{Para direccionar una carpeta de
trabajo}\label{para-direccionar-una-carpeta-de-trabajo}}

\begin{itemize}
\tightlist
\item
  Se puede utilizar las instrucciones: \texttt{cd}, \texttt{dir}.
\item
  Se puede usar la instrucción \texttt{path}. Por ejemplo:
  \texttt{/home/user/Projects/MATLAB}.
\end{itemize}

    \hypertarget{creando-un-archivo-de-variables-inicializados}{%
\subsection{Creando un archivo de variables
inicializados}\label{creando-un-archivo-de-variables-inicializados}}

    \begin{Verbatim}[commandchars=\\\{\}]
{\color{incolor}In [{\color{incolor}2}]:} \PY{c}{\PYZpc{} datos.m}
        \PY{c}{\PYZpc{} Archivo de variables inicializadas}
        \PY{n}{a} \PY{p}{=} \PY{p}{[}\PY{l+m+mi}{2} \PY{l+m+mi}{3}\PY{p}{;} \PY{l+m+mi}{4} \PY{l+m+mi}{5}\PY{p}{]}\PY{p}{;}
        \PY{n}{b} \PY{p}{=} \PY{p}{[}\PY{l+m+mi}{2} \PY{l+m+mi}{3}\PY{p}{]}\PY{o}{\PYZsq{}}\PY{p}{;}
\end{Verbatim}


    \hypertarget{para-cargar-los-datos}{%
\subsection{Para cargar los datos}\label{para-cargar-los-datos}}

    \begin{Verbatim}[commandchars=\\\{\}]
{\color{incolor}In [{\color{incolor}3}]:} \PY{c}{\PYZpc{} datos}
        \PY{n}{a}
        \PY{n}{b}
\end{Verbatim}


    \begin{Verbatim}[commandchars=\\\{\}]

a =

     2     3
     4     5


b =

     2
     3


    \end{Verbatim}

    \hypertarget{instrucciones-de-control}{%
\subsection{Instrucciones de control}\label{instrucciones-de-control}}

\hypertarget{instrucciuxf3n-for-...-end}{%
\subsubsection{\texorpdfstring{Instrucción
\texttt{for\ ...\ end}}{Instrucción for ... end}}\label{instrucciuxf3n-for-...-end}}

Sintáxis:

\begin{Shaded}
\begin{Highlighting}[]
\NormalTok{for NombVar = expresión}
\NormalTok{    instrucción_1}
\NormalTok{    instrucción_2}
\NormalTok{    ...}
\NormalTok{    instrucción_M}
\NormalTok{end}
\end{Highlighting}
\end{Shaded}

o

\begin{Shaded}
\begin{Highlighting}[]
\NormalTok{for NombVar = expresión, instrucción_1, instrucción_2, ..., instrucción_M end}
\end{Highlighting}
\end{Shaded}

donde:

\texttt{NombVar}: es el identificador de la variable.

\texttt{expresión}: usualmente tiene la siguiente forma:
\texttt{Val\_Inic:incr:Val\_Fin} o \texttt{Val\_Inic:Val\_Fin}.

\texttt{Val\_Inic}: valor inicial que a tomar la variable.

\texttt{incr}: incremento

\texttt{Val\_Fin}: valor final que va a tomar la variable.

si expresión tiene la forma \texttt{Val\_Inic:Val\_Fin}, el incremento
(implícito) es \textbf{uno}.

    A continuación, se muestra un ejemplo ilustrativo: 

    Ejemplo: Escriba una función \texttt{comb(n,k)} en MATLAB para calcular
\(\dbinom{n}{k}=\dfrac{n!}{k!(n-k)!}\).

\begin{Shaded}
\begin{Highlighting}[]
\NormalTok{function z = comb(n, k)}
\NormalTok{    z = }\FloatTok{1}\NormalTok{;}
\NormalTok{    for i=}\FloatTok{0}\NormalTok{:(k-}\FloatTok{1}\NormalTok{);}
\NormalTok{        z = z * (n-i)/(k-i);}
\NormalTok{    end}
\NormalTok{end}
\end{Highlighting}
\end{Shaded}

Guardar en el directorio de trabajo con el nombre \texttt{comb.m}. Para
ejecutar esta función, ingrese \texttt{n=4} y \texttt{k=3}, ingrese:

\begin{Shaded}
\begin{Highlighting}[]
\NormalTok{comb(}\FloatTok{4}\NormalTok{,}\FloatTok{3}\NormalTok{)}
\end{Highlighting}
\end{Shaded}

    \begin{Verbatim}[commandchars=\\\{\}]
{\color{incolor}In [{\color{incolor}4}]:} \PY{c}{\PYZpc{}\PYZpc{}file comb.m}
        \PY{k}{function}\PY{+w}{ }z \PY{p}{=}\PY{+w}{ }\PY{n+nf}{comb}\PY{p}{(}n, k\PY{p}{)}
        \PY{+w}{    }\PY{n}{z} \PY{p}{=} \PY{l+m+mi}{1}\PY{p}{;}
            \PY{k}{for} \PY{n+nb}{i} \PY{p}{=} \PY{l+m+mi}{0}\PY{p}{:}\PY{p}{(}\PY{n}{k}\PY{o}{\PYZhy{}}\PY{l+m+mi}{1}\PY{p}{)}
                \PY{n}{z} \PY{p}{=} \PY{n}{z} \PY{o}{*} \PY{p}{(}\PY{n}{n}\PY{o}{\PYZhy{}}\PY{n+nb}{i}\PY{p}{)}\PY{o}{/}\PY{p}{(}\PY{n}{k}\PY{o}{\PYZhy{}}\PY{n+nb}{i}\PY{p}{)}\PY{p}{;}
            \PY{k}{end}
        \PY{k}{end}
\end{Verbatim}


    \begin{Verbatim}[commandchars=\\\{\}]
Created file '/home/carlosal1015/GitHub/Finite-element-method-FEM/2018/Notebooks/comb.m'.

    \end{Verbatim}

    \begin{Verbatim}[commandchars=\\\{\}]
{\color{incolor}In [{\color{incolor}5}]:} \PY{n}{n} \PY{p}{=} \PY{l+m+mi}{4}\PY{p}{;} \PY{n}{k} \PY{p}{=} \PY{l+m+mi}{3}\PY{p}{;}
        \PY{n}{comb}\PY{p}{(}\PY{n}{n}\PY{p}{,} \PY{n}{k}\PY{p}{)}
\end{Verbatim}


    \begin{Verbatim}[commandchars=\\\{\}]

ans =

     4


    \end{Verbatim}

    Ejemplo: Escribir los cuadrados de los números naturales pares menores
que \(20\).

\begin{Shaded}
\begin{Highlighting}[]
\NormalTok{for i = }\FloatTok{2}\NormalTok{:}\FloatTok{2}\NormalTok{:}\FloatTok{20}
\NormalTok{    disp(i^}\FloatTok{2}\NormalTok{)}
\NormalTok{end}
\end{Highlighting}
\end{Shaded}

    \begin{Verbatim}[commandchars=\\\{\}]
{\color{incolor}In [{\color{incolor}6}]:} \PY{k}{for} \PY{n+nb}{i} \PY{p}{=} \PY{l+m+mi}{2}\PY{p}{:}\PY{l+m+mi}{2}\PY{p}{:}\PY{l+m+mi}{19}
            \PY{n+nb}{disp}\PY{p}{(}\PY{n+nb}{i}\PYZca{}\PY{l+m+mi}{2}\PY{p}{)}
        \PY{k}{end}
\end{Verbatim}


    \begin{Verbatim}[commandchars=\\\{\}]
     4

    16

    36

    64

   100

   144

   196

   256

   324


    \end{Verbatim}

    

    Aplicando el teorema del valor intermedio se obtiene el error: \[
\int_{a}^{b}\!\!f(x)\,\mathrm{d}x-T_n(f)=-\frac{h^2(b-a)}{12}f^{\prime\prime}\left(\xi\right),\quad h=\frac{b-a}{n}.
\]

    Ejemplo: Método trapezoidal compuesta

\begin{Shaded}
\begin{Highlighting}[]
\NormalTok{function inttrapS = int_trap(f, a, b, n)}
\CommentTok{% Regla trapezoidal compuesta}
\NormalTok{    h = (b-a)/n;}
\NormalTok{    t = (f(a) + f(b))/}\FloatTok{2}\NormalTok{:}
\NormalTok{    for i = }\FloatTok{1}\NormalTok{:n-}\FloatTok{1}
\NormalTok{        t = t + f(a + i*h);}
\NormalTok{    end}
\NormalTok{    inttrapS = h * t}
\NormalTok{end}
\end{Highlighting}
\end{Shaded}

    \hypertarget{regla-de-simpson}{%
\subsection{Regla de Simpson}\label{regla-de-simpson}}

\hypertarget{regla-de-simpson-13-o-regla-de-simpson-n2}{%
\subsubsection{\texorpdfstring{Regla de Simpson \(1/3\) o Regla de
Simpson
\((n=2)\):}{Regla de Simpson 1/3 o Regla de Simpson (n=2):}}\label{regla-de-simpson-13-o-regla-de-simpson-n2}}

\[
\begin{align*}
K_2(f)&=\frac{b-a}{6}\left[f(a) + 4f\left(\frac{a+b}{2}\right) + f(b)\right].\\
E_2(f)&=-\frac{h^5}{90}f^{(4)}\left(\xi\right)=-\frac{1}{2880}{\left(b-a\right)}^{5}f^{(4)}\left(\xi\right),\quad a<\xi<b.
\end{align*}
\] Si \(h=\dfrac{b-a}{2}\), \(x_0=a\), \(x_1=x_0+h\), \(x_2=x_1+h=b\),
se tiene \[
K_2(f)=\frac{1}{3}h\left(f(x_0)+4f(x_1)+f(x_2)\right)
\]

    Ejemplo: Calcule la integral \(I=\int_{I}x^2+y^2\mathrm{d}x\mathrm{d}y\)
sobre el rectángulo \(I=\left[-3,3\right]\times\left[-5,5\right]\)
usando la función \texttt{trapz} proporcionada por MATLAB.

    \begin{Verbatim}[commandchars=\\\{\}]
{\color{incolor}In [{\color{incolor}7}]:} \PY{c}{\PYZpc{} https://www.mathworks.com/help/matlab/ref/trapz.html}
        \PY{n}{x} \PY{p}{=} \PY{o}{\PYZhy{}}\PY{l+m+mi}{3}\PY{p}{:}\PY{p}{.}\PY{l+m+mi}{1}\PY{p}{:}\PY{l+m+mi}{3}\PY{p}{;} 
        \PY{n}{y} \PY{p}{=} \PY{o}{\PYZhy{}}\PY{l+m+mi}{5}\PY{p}{:}\PY{p}{.}\PY{l+m+mi}{1}\PY{p}{:}\PY{l+m+mi}{5}\PY{p}{;} 
        \PY{p}{[}\PY{n}{X}\PY{p}{,}\PY{n}{Y}\PY{p}{]} \PY{p}{=} \PY{n+nb}{meshgrid}\PY{p}{(}\PY{n}{x}\PY{p}{,}\PY{n}{y}\PY{p}{)}\PY{p}{;}
        \PY{n}{F} \PY{p}{=} \PY{n}{X}\PY{o}{.\PYZca{}}\PY{l+m+mi}{2} \PY{o}{+} \PY{n}{Y}\PY{o}{.\PYZca{}}\PY{l+m+mi}{2}\PY{p}{;}
        \PY{n}{I} \PY{p}{=} \PY{n}{trapz}\PY{p}{(}\PY{n}{y}\PY{p}{,}\PY{n}{trapz}\PY{p}{(}\PY{n}{x}\PY{p}{,}\PY{n}{F}\PY{p}{,}\PY{l+m+mi}{2}\PY{p}{)}\PY{p}{)}
\end{Verbatim}


    \begin{Verbatim}[commandchars=\\\{\}]

I =

  680.2000


    \end{Verbatim}

    

    \hypertarget{regla-compuesta-de-simpson}{%
\subsection{Regla compuesta de
Simpson}\label{regla-compuesta-de-simpson}}

De manera similar se obtiene la \emph{Regla Compuesta de Simpson}. Sean
\(n=2m\) (par), \(m\in\mathbb{N}\), \(h=\dfrac{(b-a)}{n}\),
\(x_j=a+jh\), \(j\in\left\{0,1,2,\ldots,n\right\}\) y
\(f\in\mathcal{C}^{4}\left[a,b\right]\).

Simplificando la regla de Simpson en los subintervalos
\(\left[x_{j-1},x_j\right]\) con \(j\in\left\{1,2,\ldots, n\right\}\) se
obtiene

\[
\int_{a}^{b}\!\!f(x)\,\mathrm{d}x=S_n\left[a,b\right]-\frac{h^4\left(b-a\right)}{180}\left(\frac{1}{m}\sum_{j=1}^{m}f^{(4)}\left(\xi_j\right)\right),\quad \xi_j\in\left]x_{j-1},x_j\right[.
\]

    Ejemplo: Método Compuesta de Simpson (1/3)

\begin{Shaded}
\begin{Highlighting}[]
\NormalTok{function intsimpS = int_simp(f, a, b, n)}
\CommentTok{% Se aplica M = 2*n veces el método de Simpson}
\CommentTok{% n+1 puntos de partición}
\NormalTok{    M = n/}\FloatTok{2}\NormalTok{; s = }\FloatTok{0}\NormalTok{; h =(b-a)/n;}
\NormalTok{    s1 = f(a+h * (}\FloatTok{2}\NormalTok{*M-}\FloatTok{1}\NormalTok{)); s2 = }\FloatTok{0}\NormalTok{;}
\NormalTok{    for j=}\FloatTok{1}\NormalTok{: M-}\FloatTok{1}
\NormalTok{        s1 = s1 + f(a + h*(}\FloatTok{2}\NormalTok{*j-}\FloatTok{1}\NormalTok{));}
\NormalTok{        s2 = s2 + f(a + h*(}\FloatTok{2}\NormalTok{*j));}
\NormalTok{    end}
\NormalTok{    intsimpS = h * (f(a) + }\FloatTok{4}\NormalTok{*s1+}\FloatTok{2}\NormalTok{*s2+f(b))/}\FloatTok{3}\NormalTok{;}
\NormalTok{end}
\end{Highlighting}
\end{Shaded}

    Ejemplo: Calcule la siguiente sumatoria: \[
S=\sum_{k=0}^{m}\frac{k^k}{k!} \text{ para } x=5 \text{ y } m = 24.
\]

    \begin{Verbatim}[commandchars=\\\{\}]
{\color{incolor}In [{\color{incolor}8}]:} \PY{n}{x} \PY{p}{=} \PY{l+m+mi}{5}\PY{p}{;} \PY{n}{m} \PY{p}{=} \PY{l+m+mi}{24}\PY{p}{;}
        \PY{n}{sum} \PY{p}{=} \PY{l+m+mi}{1}\PY{p}{;} \PY{n}{fact} \PY{p}{=} \PY{l+m+mi}{1}\PY{p}{;}
        \PY{k}{for} \PY{n}{k}\PY{p}{=}\PY{l+m+mi}{1}\PY{p}{:}\PY{n}{m}
            \PY{n}{fact} \PY{p}{=} \PY{n}{fact} \PY{o}{*} \PY{n}{x}\PY{o}{/}\PY{n}{k}\PY{p}{;}
            \PY{n}{sum} \PY{p}{=} \PY{n}{sum} \PY{o}{+} \PY{n}{fact}\PY{p}{;}
        \PY{k}{end}
        \PY{n+nb}{disp}\PY{p}{(}\PY{n}{sum}\PY{p}{)}
        \PY{c}{\PYZpc{} Verificando: El resultado debe acercase a cero.}
        \PY{n+nb}{disp}\PY{p}{(}\PY{n}{sum} \PY{o}{\PYZhy{}} \PY{n+nb}{exp}\PY{p}{(}\PY{l+m+mi}{5}\PY{p}{)}\PY{p}{)}
\end{Verbatim}


    \begin{Verbatim}[commandchars=\\\{\}]
  148.4132

  -2.3740e-08


    \end{Verbatim}

    \hypertarget{instrucciuxf3n-if}{%
\subsection{\texorpdfstring{Instrucción
\texttt{if}}{Instrucción if}}\label{instrucciuxf3n-if}}

Sintáxis:

\begin{Shaded}
\begin{Highlighting}[]
\NormalTok{if relación}
\NormalTok{    instrucciones}
\NormalTok{end}
\end{Highlighting}
\end{Shaded}

Las instrucciones se ejecutarán solo si la relación es cierta.

La forma general de la sentencia \texttt{if} es

\begin{Shaded}
\begin{Highlighting}[]
\NormalTok{if expresión1}
\NormalTok{    sentencias1}
\NormalTok{elseif expresión2}
\NormalTok{    sentencias2}
\NormalTok{else}
\NormalTok{    sentencias3}
\NormalTok{end}
\end{Highlighting}
\end{Shaded}

Las sentencias serán ejecutadas si la parte real de la expresión tiene
todos los elementos no ceros.

La partes \texttt{else} y \texttt{elseif} son opcionales.

La expresión usualmente tiene la forma \texttt{expr\ oper\ expr} donde
\texttt{oper} es: \texttt{==}, \texttt{\textless{}},
\texttt{\textgreater{}}, \texttt{\textless{}=},
\texttt{\textgreater{}=}, \texttt{\textasciitilde{}=}.

Ejemplo:

    \begin{Verbatim}[commandchars=\\\{\}]
{\color{incolor}In [{\color{incolor}9}]:} \PY{c}{\PYZpc{} Recuerde que n vale 4 :\PYZhy{})}
        \PY{k}{if} \PY{n}{n} \PY{o}{\PYZlt{}} \PY{l+m+mi}{0}
            \PY{n}{paridad} \PY{p}{=} \PY{l+m+mi}{0}\PY{p}{;}
        \PY{k}{elseif} \PY{n+nb}{rem}\PY{p}{(}\PY{n}{n}\PY{p}{,}\PY{l+m+mi}{2}\PY{p}{)} \PY{o}{==} \PY{l+m+mi}{0}
            \PY{n}{paridad} \PY{p}{=} \PY{l+m+mi}{2}\PY{p}{;}
        \PY{k}{else}
            \PY{n}{paridad} \PY{p}{=} \PY{l+m+mi}{1}\PY{p}{;}
        \PY{k}{end}
        \PY{n+nb}{disp}\PY{p}{(}\PY{n}{paridad}\PY{p}{)}
\end{Verbatim}


    \begin{Verbatim}[commandchars=\\\{\}]
     2


    \end{Verbatim}

    \hypertarget{relaciones}{%
\subsection{Relaciones:}\label{relaciones}}

Los operadores relacionales en MATLAB son

\begin{longtable}[]{@{}cc@{}}
\toprule
Operador relacional & Significado\tabularnewline
\midrule
\endhead
\texttt{\textless{}} & menor que\tabularnewline
\texttt{\textgreater{}} & mayor que\tabularnewline
\texttt{\textless{}=} & menor o igual que\tabularnewline
\texttt{\textgreater{}=} & mayor o igual que\tabularnewline
\texttt{==} & igual\tabularnewline
\texttt{\textasciitilde{}=} & no igual\tabularnewline
\bottomrule
\end{longtable}

Los operadores pueden conectarse o cuantificarse por los operadores
lógicos

\begin{longtable}[]{@{}cc@{}}
\toprule
Operador lógico & Significado\tabularnewline
\midrule
\endhead
\texttt{\&} & y (conjunción)\tabularnewline
pipe o tubería & o (disyunción inclusiva)\tabularnewline
\texttt{\textasciitilde{}} & no (negación)\tabularnewline
\bottomrule
\end{longtable}

Cuando se aplican a escalares, la relación es realmente el escalar
\texttt{1} o \texttt{0} dependiendo de si la relación es
\emph{verdadera} o \emph{falsa}.

Ejemplo:

    \begin{Verbatim}[commandchars=\\\{\}]
{\color{incolor}In [{\color{incolor}10}]:} \PY{l+m+mi}{4} \PY{o}{\PYZlt{}} \PY{l+m+mi}{6}
         \PY{l+m+mi}{3} \PY{o}{\PYZgt{}} \PY{l+m+mi}{8}
         \PY{l+m+mi}{4} \PY{o}{==} \PY{l+m+mi}{4}
         \PY{l+m+mi}{3} \PY{o}{==} \PY{l+m+mi}{6}
\end{Verbatim}


    \begin{Verbatim}[commandchars=\\\{\}]

ans =

  logical

   1


ans =

  logical

   0


ans =

  logical

   1


ans =

  logical

   0


    \end{Verbatim}

    Ejemplo: Escriba una función que encuentre el mayor de dos números
\texttt{a}, \texttt{b}.

\begin{Shaded}
\begin{Highlighting}[]
\NormalTok{function SalMay = mayor(a, b)}
\NormalTok{if a > b}
\NormalTok{    SalMay = a;}
\NormalTok{else}
\NormalTok{    SalMay = b;}
\NormalTok{end}
\end{Highlighting}
\end{Shaded}

    \begin{Verbatim}[commandchars=\\\{\}]
{\color{incolor}In [{\color{incolor}11}]:} \PY{c}{\PYZpc{}\PYZpc{}file mayor.m}
         \PY{k}{function}\PY{+w}{ }SalMay \PY{p}{=}\PY{+w}{ }\PY{n+nf}{mayor}\PY{p}{(}a, b\PY{p}{)}
         \PY{k}{if} \PY{n}{a} \PY{o}{\PYZgt{}} \PY{n}{b}
             \PY{n}{SalMay} \PY{p}{=} \PY{n}{a}\PY{p}{;}
         \PY{k}{else}
             \PY{n}{SalMay} \PY{p}{=} \PY{n}{b}\PY{p}{;}
         \PY{k}{end}
\end{Verbatim}


    \begin{Verbatim}[commandchars=\\\{\}]
Created file '/home/carlosal1015/GitHub/Finite-element-method-FEM/2018/Notebooks/mayor.m'.

    \end{Verbatim}

    \begin{Verbatim}[commandchars=\\\{\}]
{\color{incolor}In [{\color{incolor}12}]:} \PY{n}{mayor}\PY{p}{(}\PY{l+m+mi}{4}\PY{p}{,}\PY{l+m+mi}{5}\PY{p}{)}
\end{Verbatim}


    \begin{Verbatim}[commandchars=\\\{\}]

ans =

     5


    \end{Verbatim}

    \begin{Verbatim}[commandchars=\\\{\}]
{\color{incolor}In [{\color{incolor}13}]:} \PY{c}{\PYZpc{}\PYZpc{}file grafica\PYZus{}hat.m}
         \PY{k}{function}\PY{+w}{ }\PY{n+nf}{grafica\PYZus{}hat}\PY{p}{(}j, a, b, n, s\PY{p}{)}
         \PY{c}{\PYZpc{} n segmentos}
         \PY{n}{x} \PY{p}{=} \PY{n+nb}{linspace}\PY{p}{(}\PY{n}{a}\PY{p}{,} \PY{n}{b}\PY{p}{,} \PY{n}{n} \PY{o}{+} \PY{l+m+mi}{1}\PY{p}{)}\PY{p}{;}
         \PY{k}{if} \PY{n+nb}{j} \PY{o}{==} \PY{l+m+mi}{1}
             \PY{n}{xx} \PY{p}{=} \PY{p}{[}\PY{n}{x}\PY{p}{(}\PY{l+m+mi}{1}\PY{p}{)}\PY{p}{,} \PY{n}{x}\PY{p}{(}\PY{l+m+mi}{2}\PY{p}{)}\PY{p}{,} \PY{n}{x}\PY{p}{(}\PY{n}{n} \PY{o}{+} \PY{l+m+mi}{1}\PY{p}{)}\PY{p}{]}\PY{p}{;}
             \PY{n}{yy} \PY{p}{=} \PY{p}{[}\PY{l+m+mi}{1}\PY{p}{,} \PY{l+m+mi}{0}\PY{p}{,} \PY{l+m+mi}{0}\PY{p}{]}\PY{p}{;}
         \PY{k}{else}
             \PY{k}{if} \PY{n+nb}{j} \PY{o}{==} \PY{n}{n}
                 \PY{n}{xx} \PY{p}{=} \PY{p}{[}\PY{n}{x}\PY{p}{(}\PY{l+m+mi}{1}\PY{p}{)}\PY{p}{,} \PY{n}{x}\PY{p}{(}\PY{n}{n}\PY{p}{)}\PY{p}{,} \PY{n}{x}\PY{p}{(}\PY{n}{n} \PY{o}{+} \PY{l+m+mi}{1}\PY{p}{)}\PY{p}{]}\PY{p}{;}
                 \PY{n}{yy} \PY{p}{=} \PY{p}{[}\PY{l+m+mi}{0}\PY{p}{,} \PY{l+m+mi}{0}\PY{p}{,} \PY{l+m+mi}{1}\PY{p}{]}\PY{p}{;}
             \PY{k}{else}
                 \PY{n}{xx} \PY{p}{=} \PY{p}{[}\PY{n}{x}\PY{p}{(}\PY{l+m+mi}{1}\PY{p}{)}\PY{p}{,} \PY{n}{x}\PY{p}{(}\PY{n+nb}{j} \PY{o}{\PYZhy{}} \PY{l+m+mi}{1}\PY{p}{)}\PY{p}{,} \PY{n}{x}\PY{p}{(}\PY{n+nb}{j}\PY{p}{)}\PY{p}{,} \PY{n}{x}\PY{p}{(}\PY{n+nb}{j} \PY{o}{+} \PY{l+m+mi}{1}\PY{p}{)}\PY{p}{,} \PY{n}{x}\PY{p}{(}\PY{n}{n} \PY{o}{+} \PY{l+m+mi}{1}\PY{p}{)}\PY{p}{]}\PY{p}{;}
                 \PY{n}{yy} \PY{p}{=} \PY{p}{[}\PY{l+m+mi}{0}\PY{p}{,} \PY{l+m+mi}{0}\PY{p}{,} \PY{l+m+mi}{1}\PY{p}{,} \PY{l+m+mi}{0}\PY{p}{,} \PY{l+m+mi}{0}\PY{p}{]}\PY{p}{;}
             \PY{k}{end}
         \PY{k}{end}
         \PY{n}{plot}\PY{p}{(}\PY{n}{xx}\PY{p}{,} \PY{n}{yy}\PY{p}{,} \PY{n}{s}\PY{p}{)}
         \PY{n}{shg}
         \PY{k}{end}
\end{Verbatim}


    \begin{Verbatim}[commandchars=\\\{\}]
Created file '/home/carlosal1015/GitHub/Finite-element-method-FEM/2018/Notebooks/grafica\_hat.m'.

    \end{Verbatim}

    \begin{Verbatim}[commandchars=\\\{\}]
{\color{incolor}In [{\color{incolor}14}]:} \PY{c}{\PYZpc{}\PYZpc{}file PL2.m}
         \PY{k}{function}\PY{+w}{ }[x, proy] \PY{p}{=}\PY{+w}{ }\PY{n+nf}{PL2}\PY{p}{(}f, a, b, n\PY{p}{)}
         \PY{+w}{    }\PY{n}{M} \PY{p}{=} \PY{n+nb}{zeros}\PY{p}{(}\PY{n}{n} \PY{o}{+} \PY{l+m+mi}{1}\PY{p}{,} \PY{n}{n} \PY{o}{+} \PY{l+m+mi}{1}\PY{p}{)}\PY{p}{;}
             \PY{n}{c} \PY{p}{=} \PY{n+nb}{zeros}\PY{p}{(}\PY{n}{n} \PY{o}{+} \PY{l+m+mi}{1}\PY{p}{,} \PY{l+m+mi}{1}\PY{p}{)}\PY{p}{;}
             \PY{n}{h} \PY{p}{=} \PY{p}{(}\PY{n}{b}\PY{o}{\PYZhy{}}\PY{n}{a}\PY{p}{)}\PY{o}{/}\PY{n}{n}\PY{p}{;}
             \PY{n}{x} \PY{p}{=} \PY{n+nb}{linspace}\PY{p}{(}\PY{n}{a}\PY{p}{,} \PY{n}{b}\PY{p}{,} \PY{n}{n} \PY{o}{+} \PY{l+m+mi}{1}\PY{p}{)}\PY{p}{;}
             
             \PY{k}{for} \PY{n}{k} \PY{p}{=} \PY{l+m+mi}{1}\PY{p}{:}\PY{n}{n}
                 \PY{n}{M\PYZus{}local} \PY{p}{=} \PY{n}{h} \PY{o}{*} \PY{p}{[}\PY{l+m+mi}{1}\PY{o}{/}\PY{l+m+mi}{3} \PY{l+m+mi}{1}\PY{o}{/}\PY{l+m+mi}{6}\PY{p}{;} \PY{l+m+mi}{1}\PY{o}{/}\PY{l+m+mi}{6} \PY{l+m+mi}{1}\PY{o}{/}\PY{l+m+mi}{3}\PY{p}{]}\PY{p}{;}
                 \PY{n}{b\PYZus{}local} \PY{p}{=} \PY{p}{(}\PY{n}{h}\PY{o}{/}\PY{l+m+mi}{2}\PY{p}{)} \PY{o}{*} \PY{p}{[} \PY{n}{f}\PY{p}{(}\PY{n}{x}\PY{p}{(}\PY{n}{k}\PY{p}{)}\PY{p}{)}\PY{p}{;} \PY{n}{f}\PY{p}{(}\PY{n}{x}\PY{p}{(}\PY{n}{k}\PY{o}{+}\PY{l+m+mi}{1}\PY{p}{)}\PY{p}{)} \PY{p}{]}\PY{p}{;}
                 \PY{n}{M}\PY{p}{(}\PY{n}{k}\PY{p}{:}\PY{n}{k} \PY{o}{+} \PY{l+m+mi}{1}\PY{p}{,} \PY{n}{k}\PY{p}{:}\PY{n}{k} \PY{o}{+} \PY{l+m+mi}{1}\PY{p}{)} \PY{p}{=} \PY{n}{M\PYZus{}local} \PY{o}{+} \PY{n}{M}\PY{p}{(}\PY{n}{k}\PY{p}{:}\PY{n}{k} \PY{o}{+} \PY{l+m+mi}{1}\PY{p}{,} \PY{n}{k}\PY{p}{:}\PY{n}{k} \PY{o}{+} \PY{l+m+mi}{1}\PY{p}{)}\PY{p}{;}
                 \PY{n}{c}\PY{p}{(}\PY{n}{k}\PY{p}{:}\PY{n}{k} \PY{o}{+} \PY{l+m+mi}{1}\PY{p}{,} \PY{l+m+mi}{1}\PY{p}{)} \PY{p}{=} \PY{n}{c}\PY{p}{(}\PY{n}{k}\PY{p}{:}\PY{n}{k} \PY{o}{+} \PY{l+m+mi}{1}\PY{p}{,} \PY{l+m+mi}{1}\PY{p}{)} \PY{o}{+} \PY{n}{b\PYZus{}local}\PY{p}{;}
             \PY{k}{end}
         \PY{n}{proy} \PY{p}{=} \PY{n}{M}\PY{o}{\PYZbs{}}\PY{n}{c}\PY{p}{;}
         \PY{c}{\PYZpc{}display(M)}
         \PY{c}{\PYZpc{}display(c)}
         \PY{k}{end}
\end{Verbatim}


    \begin{Verbatim}[commandchars=\\\{\}]
Created file '/home/carlosal1015/GitHub/Finite-element-method-FEM/2018/Notebooks/PL2.m'.

    \end{Verbatim}

    \begin{Verbatim}[commandchars=\\\{\}]
{\color{incolor}In [{\color{incolor}15}]:} \PY{c}{\PYZpc{} Corrida de la Proyección L2 de una función}
         \PY{c}{\PYZpc{} Grabe: run\PYZus{}PL2.m}
         \PY{n}{a} \PY{p}{=} \PY{l+m+mi}{0}\PY{p}{;}
         \PY{n}{b} \PY{p}{=} \PY{l+m+mi}{1}\PY{p}{;}
         \PY{n}{n} \PY{p}{=} \PY{l+m+mi}{250}\PY{p}{;} \PY{c}{\PYZpc{} n puede tomar valores de 100, 250, etc.}
         \PY{c}{\PYZpc{} f=@(x) 2*x.*sin(2*pi*x)+3;}
         \PY{c}{\PYZpc{} f=@(x) 0*x+1;}
         \PY{c}{\PYZpc{} f=@(x) x.\PYZca{}3.*(x\PYZhy{}1).*(1\PYZhy{}2*x);}
         \PY{n}{epsilon} \PY{p}{=} \PY{l+m+mf}{0.01}\PY{p}{;}
         \PY{n}{f} \PY{p}{=} \PY{p}{@}\PY{p}{(}\PY{n}{x}\PY{p}{)} \PY{n+nb}{atan}\PY{p}{(} \PY{p}{(}\PY{n}{x}\PY{o}{\PYZhy{}}\PY{l+m+mf}{0.5}\PY{p}{)}\PY{o}{/}\PY{n}{epsilon} \PY{p}{)}\PY{p}{;}
         
         \PY{p}{[}\PY{n}{x}\PY{p}{,} \PY{n}{proy}\PY{p}{]} \PY{p}{=} \PY{n}{PL2}\PY{p}{(}\PY{n}{f}\PY{p}{,} \PY{n}{a}\PY{p}{,} \PY{n}{b}\PY{p}{,} \PY{n}{n}\PY{p}{)}\PY{p}{;}
         \PY{n}{hold} \PY{n}{on}\PY{p}{;}
         \PY{n}{t} \PY{p}{=} \PY{n+nb}{linspace}\PY{p}{(}\PY{n}{a}\PY{p}{,} \PY{n}{b}\PY{p}{,} \PY{l+m+mi}{250}\PY{p}{)}\PY{p}{;}
         \PY{n}{plot}\PY{p}{(}\PY{n}{t}\PY{p}{,} \PY{n}{f}\PY{p}{(}\PY{n}{t}\PY{p}{)}\PY{p}{,} \PY{l+s}{\PYZsq{}}\PY{l+s}{\PYZhy{}r\PYZsq{}}\PY{p}{,} \PY{l+s}{\PYZsq{}}\PY{l+s}{LineWidth\PYZsq{}}\PY{p}{,} \PY{l+m+mi}{1}\PY{p}{)}\PY{p}{;}
         \PY{n}{shg}\PY{p}{;}
         \PY{n}{hold} \PY{n}{off}
\end{Verbatim}


    \begin{center}
    \adjustimage{max size={0.9\linewidth}{0.9\paperheight}}{output_36_0.png}
    \end{center}
    { \hspace*{\fill} \\}
    

    % Add a bibliography block to the postdoc
    
    
    
    \end{document}
