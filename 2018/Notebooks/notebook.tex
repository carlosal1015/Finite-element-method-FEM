
% Default to the notebook output style

    


% Inherit from the specified cell style.




    
\documentclass[11pt]{article}

    
    
    \usepackage[T1]{fontenc}
    % Nicer default font (+ math font) than Computer Modern for most use cases
    \usepackage{mathpazo}

    % Basic figure setup, for now with no caption control since it's done
    % automatically by Pandoc (which extracts ![](path) syntax from Markdown).
    \usepackage{graphicx}
    % We will generate all images so they have a width \maxwidth. This means
    % that they will get their normal width if they fit onto the page, but
    % are scaled down if they would overflow the margins.
    \makeatletter
    \def\maxwidth{\ifdim\Gin@nat@width>\linewidth\linewidth
    \else\Gin@nat@width\fi}
    \makeatother
    \let\Oldincludegraphics\includegraphics
    % Set max figure width to be 80% of text width, for now hardcoded.
    \renewcommand{\includegraphics}[1]{\Oldincludegraphics[width=.8\maxwidth]{#1}}
    % Ensure that by default, figures have no caption (until we provide a
    % proper Figure object with a Caption API and a way to capture that
    % in the conversion process - todo).
    \usepackage{caption}
    \DeclareCaptionLabelFormat{nolabel}{}
    \captionsetup{labelformat=nolabel}

    \usepackage{adjustbox} % Used to constrain images to a maximum size 
    \usepackage{xcolor} % Allow colors to be defined
    \usepackage{enumerate} % Needed for markdown enumerations to work
    \usepackage{geometry} % Used to adjust the document margins
    \usepackage{amsmath} % Equations
    \usepackage{amssymb} % Equations
    \usepackage{textcomp} % defines textquotesingle
    % Hack from http://tex.stackexchange.com/a/47451/13684:
    \AtBeginDocument{%
        \def\PYZsq{\textquotesingle}% Upright quotes in Pygmentized code
    }
    \usepackage{upquote} % Upright quotes for verbatim code
    \usepackage{eurosym} % defines \euro
    \usepackage[mathletters]{ucs} % Extended unicode (utf-8) support
    \usepackage[utf8x]{inputenc} % Allow utf-8 characters in the tex document
    \usepackage{fancyvrb} % verbatim replacement that allows latex
    \usepackage{grffile} % extends the file name processing of package graphics 
                         % to support a larger range 
    % The hyperref package gives us a pdf with properly built
    % internal navigation ('pdf bookmarks' for the table of contents,
    % internal cross-reference links, web links for URLs, etc.)
    \usepackage{hyperref}
    \usepackage{longtable} % longtable support required by pandoc >1.10
    \usepackage{booktabs}  % table support for pandoc > 1.12.2
    \usepackage[inline]{enumitem} % IRkernel/repr support (it uses the enumerate* environment)
    \usepackage[normalem]{ulem} % ulem is needed to support strikethroughs (\sout)
                                % normalem makes italics be italics, not underlines
    

    
    
    % Colors for the hyperref package
    \definecolor{urlcolor}{rgb}{0,.145,.698}
    \definecolor{linkcolor}{rgb}{.71,0.21,0.01}
    \definecolor{citecolor}{rgb}{.12,.54,.11}

    % ANSI colors
    \definecolor{ansi-black}{HTML}{3E424D}
    \definecolor{ansi-black-intense}{HTML}{282C36}
    \definecolor{ansi-red}{HTML}{E75C58}
    \definecolor{ansi-red-intense}{HTML}{B22B31}
    \definecolor{ansi-green}{HTML}{00A250}
    \definecolor{ansi-green-intense}{HTML}{007427}
    \definecolor{ansi-yellow}{HTML}{DDB62B}
    \definecolor{ansi-yellow-intense}{HTML}{B27D12}
    \definecolor{ansi-blue}{HTML}{208FFB}
    \definecolor{ansi-blue-intense}{HTML}{0065CA}
    \definecolor{ansi-magenta}{HTML}{D160C4}
    \definecolor{ansi-magenta-intense}{HTML}{A03196}
    \definecolor{ansi-cyan}{HTML}{60C6C8}
    \definecolor{ansi-cyan-intense}{HTML}{258F8F}
    \definecolor{ansi-white}{HTML}{C5C1B4}
    \definecolor{ansi-white-intense}{HTML}{A1A6B2}

    % commands and environments needed by pandoc snippets
    % extracted from the output of `pandoc -s`
    \providecommand{\tightlist}{%
      \setlength{\itemsep}{0pt}\setlength{\parskip}{0pt}}
    \DefineVerbatimEnvironment{Highlighting}{Verbatim}{commandchars=\\\{\}}
    % Add ',fontsize=\small' for more characters per line
    \newenvironment{Shaded}{}{}
    \newcommand{\KeywordTok}[1]{\textcolor[rgb]{0.00,0.44,0.13}{\textbf{{#1}}}}
    \newcommand{\DataTypeTok}[1]{\textcolor[rgb]{0.56,0.13,0.00}{{#1}}}
    \newcommand{\DecValTok}[1]{\textcolor[rgb]{0.25,0.63,0.44}{{#1}}}
    \newcommand{\BaseNTok}[1]{\textcolor[rgb]{0.25,0.63,0.44}{{#1}}}
    \newcommand{\FloatTok}[1]{\textcolor[rgb]{0.25,0.63,0.44}{{#1}}}
    \newcommand{\CharTok}[1]{\textcolor[rgb]{0.25,0.44,0.63}{{#1}}}
    \newcommand{\StringTok}[1]{\textcolor[rgb]{0.25,0.44,0.63}{{#1}}}
    \newcommand{\CommentTok}[1]{\textcolor[rgb]{0.38,0.63,0.69}{\textit{{#1}}}}
    \newcommand{\OtherTok}[1]{\textcolor[rgb]{0.00,0.44,0.13}{{#1}}}
    \newcommand{\AlertTok}[1]{\textcolor[rgb]{1.00,0.00,0.00}{\textbf{{#1}}}}
    \newcommand{\FunctionTok}[1]{\textcolor[rgb]{0.02,0.16,0.49}{{#1}}}
    \newcommand{\RegionMarkerTok}[1]{{#1}}
    \newcommand{\ErrorTok}[1]{\textcolor[rgb]{1.00,0.00,0.00}{\textbf{{#1}}}}
    \newcommand{\NormalTok}[1]{{#1}}
    
    % Additional commands for more recent versions of Pandoc
    \newcommand{\ConstantTok}[1]{\textcolor[rgb]{0.53,0.00,0.00}{{#1}}}
    \newcommand{\SpecialCharTok}[1]{\textcolor[rgb]{0.25,0.44,0.63}{{#1}}}
    \newcommand{\VerbatimStringTok}[1]{\textcolor[rgb]{0.25,0.44,0.63}{{#1}}}
    \newcommand{\SpecialStringTok}[1]{\textcolor[rgb]{0.73,0.40,0.53}{{#1}}}
    \newcommand{\ImportTok}[1]{{#1}}
    \newcommand{\DocumentationTok}[1]{\textcolor[rgb]{0.73,0.13,0.13}{\textit{{#1}}}}
    \newcommand{\AnnotationTok}[1]{\textcolor[rgb]{0.38,0.63,0.69}{\textbf{\textit{{#1}}}}}
    \newcommand{\CommentVarTok}[1]{\textcolor[rgb]{0.38,0.63,0.69}{\textbf{\textit{{#1}}}}}
    \newcommand{\VariableTok}[1]{\textcolor[rgb]{0.10,0.09,0.49}{{#1}}}
    \newcommand{\ControlFlowTok}[1]{\textcolor[rgb]{0.00,0.44,0.13}{\textbf{{#1}}}}
    \newcommand{\OperatorTok}[1]{\textcolor[rgb]{0.40,0.40,0.40}{{#1}}}
    \newcommand{\BuiltInTok}[1]{{#1}}
    \newcommand{\ExtensionTok}[1]{{#1}}
    \newcommand{\PreprocessorTok}[1]{\textcolor[rgb]{0.74,0.48,0.00}{{#1}}}
    \newcommand{\AttributeTok}[1]{\textcolor[rgb]{0.49,0.56,0.16}{{#1}}}
    \newcommand{\InformationTok}[1]{\textcolor[rgb]{0.38,0.63,0.69}{\textbf{\textit{{#1}}}}}
    \newcommand{\WarningTok}[1]{\textcolor[rgb]{0.38,0.63,0.69}{\textbf{\textit{{#1}}}}}
    
    
    % Define a nice break command that doesn't care if a line doesn't already
    % exist.
    \def\br{\hspace*{\fill} \\* }
    % Math Jax compatability definitions
    \def\gt{>}
    \def\lt{<}
    % Document parameters
    \title{Functions\_in\_MATLAB}
    
    
    

    % Pygments definitions
    
\makeatletter
\def\PY@reset{\let\PY@it=\relax \let\PY@bf=\relax%
    \let\PY@ul=\relax \let\PY@tc=\relax%
    \let\PY@bc=\relax \let\PY@ff=\relax}
\def\PY@tok#1{\csname PY@tok@#1\endcsname}
\def\PY@toks#1+{\ifx\relax#1\empty\else%
    \PY@tok{#1}\expandafter\PY@toks\fi}
\def\PY@do#1{\PY@bc{\PY@tc{\PY@ul{%
    \PY@it{\PY@bf{\PY@ff{#1}}}}}}}
\def\PY#1#2{\PY@reset\PY@toks#1+\relax+\PY@do{#2}}

\expandafter\def\csname PY@tok@w\endcsname{\def\PY@tc##1{\textcolor[rgb]{0.73,0.73,0.73}{##1}}}
\expandafter\def\csname PY@tok@c\endcsname{\let\PY@it=\textit\def\PY@tc##1{\textcolor[rgb]{0.25,0.50,0.50}{##1}}}
\expandafter\def\csname PY@tok@cp\endcsname{\def\PY@tc##1{\textcolor[rgb]{0.74,0.48,0.00}{##1}}}
\expandafter\def\csname PY@tok@k\endcsname{\let\PY@bf=\textbf\def\PY@tc##1{\textcolor[rgb]{0.00,0.50,0.00}{##1}}}
\expandafter\def\csname PY@tok@kp\endcsname{\def\PY@tc##1{\textcolor[rgb]{0.00,0.50,0.00}{##1}}}
\expandafter\def\csname PY@tok@kt\endcsname{\def\PY@tc##1{\textcolor[rgb]{0.69,0.00,0.25}{##1}}}
\expandafter\def\csname PY@tok@o\endcsname{\def\PY@tc##1{\textcolor[rgb]{0.40,0.40,0.40}{##1}}}
\expandafter\def\csname PY@tok@ow\endcsname{\let\PY@bf=\textbf\def\PY@tc##1{\textcolor[rgb]{0.67,0.13,1.00}{##1}}}
\expandafter\def\csname PY@tok@nb\endcsname{\def\PY@tc##1{\textcolor[rgb]{0.00,0.50,0.00}{##1}}}
\expandafter\def\csname PY@tok@nf\endcsname{\def\PY@tc##1{\textcolor[rgb]{0.00,0.00,1.00}{##1}}}
\expandafter\def\csname PY@tok@nc\endcsname{\let\PY@bf=\textbf\def\PY@tc##1{\textcolor[rgb]{0.00,0.00,1.00}{##1}}}
\expandafter\def\csname PY@tok@nn\endcsname{\let\PY@bf=\textbf\def\PY@tc##1{\textcolor[rgb]{0.00,0.00,1.00}{##1}}}
\expandafter\def\csname PY@tok@ne\endcsname{\let\PY@bf=\textbf\def\PY@tc##1{\textcolor[rgb]{0.82,0.25,0.23}{##1}}}
\expandafter\def\csname PY@tok@nv\endcsname{\def\PY@tc##1{\textcolor[rgb]{0.10,0.09,0.49}{##1}}}
\expandafter\def\csname PY@tok@no\endcsname{\def\PY@tc##1{\textcolor[rgb]{0.53,0.00,0.00}{##1}}}
\expandafter\def\csname PY@tok@nl\endcsname{\def\PY@tc##1{\textcolor[rgb]{0.63,0.63,0.00}{##1}}}
\expandafter\def\csname PY@tok@ni\endcsname{\let\PY@bf=\textbf\def\PY@tc##1{\textcolor[rgb]{0.60,0.60,0.60}{##1}}}
\expandafter\def\csname PY@tok@na\endcsname{\def\PY@tc##1{\textcolor[rgb]{0.49,0.56,0.16}{##1}}}
\expandafter\def\csname PY@tok@nt\endcsname{\let\PY@bf=\textbf\def\PY@tc##1{\textcolor[rgb]{0.00,0.50,0.00}{##1}}}
\expandafter\def\csname PY@tok@nd\endcsname{\def\PY@tc##1{\textcolor[rgb]{0.67,0.13,1.00}{##1}}}
\expandafter\def\csname PY@tok@s\endcsname{\def\PY@tc##1{\textcolor[rgb]{0.73,0.13,0.13}{##1}}}
\expandafter\def\csname PY@tok@sd\endcsname{\let\PY@it=\textit\def\PY@tc##1{\textcolor[rgb]{0.73,0.13,0.13}{##1}}}
\expandafter\def\csname PY@tok@si\endcsname{\let\PY@bf=\textbf\def\PY@tc##1{\textcolor[rgb]{0.73,0.40,0.53}{##1}}}
\expandafter\def\csname PY@tok@se\endcsname{\let\PY@bf=\textbf\def\PY@tc##1{\textcolor[rgb]{0.73,0.40,0.13}{##1}}}
\expandafter\def\csname PY@tok@sr\endcsname{\def\PY@tc##1{\textcolor[rgb]{0.73,0.40,0.53}{##1}}}
\expandafter\def\csname PY@tok@ss\endcsname{\def\PY@tc##1{\textcolor[rgb]{0.10,0.09,0.49}{##1}}}
\expandafter\def\csname PY@tok@sx\endcsname{\def\PY@tc##1{\textcolor[rgb]{0.00,0.50,0.00}{##1}}}
\expandafter\def\csname PY@tok@m\endcsname{\def\PY@tc##1{\textcolor[rgb]{0.40,0.40,0.40}{##1}}}
\expandafter\def\csname PY@tok@gh\endcsname{\let\PY@bf=\textbf\def\PY@tc##1{\textcolor[rgb]{0.00,0.00,0.50}{##1}}}
\expandafter\def\csname PY@tok@gu\endcsname{\let\PY@bf=\textbf\def\PY@tc##1{\textcolor[rgb]{0.50,0.00,0.50}{##1}}}
\expandafter\def\csname PY@tok@gd\endcsname{\def\PY@tc##1{\textcolor[rgb]{0.63,0.00,0.00}{##1}}}
\expandafter\def\csname PY@tok@gi\endcsname{\def\PY@tc##1{\textcolor[rgb]{0.00,0.63,0.00}{##1}}}
\expandafter\def\csname PY@tok@gr\endcsname{\def\PY@tc##1{\textcolor[rgb]{1.00,0.00,0.00}{##1}}}
\expandafter\def\csname PY@tok@ge\endcsname{\let\PY@it=\textit}
\expandafter\def\csname PY@tok@gs\endcsname{\let\PY@bf=\textbf}
\expandafter\def\csname PY@tok@gp\endcsname{\let\PY@bf=\textbf\def\PY@tc##1{\textcolor[rgb]{0.00,0.00,0.50}{##1}}}
\expandafter\def\csname PY@tok@go\endcsname{\def\PY@tc##1{\textcolor[rgb]{0.53,0.53,0.53}{##1}}}
\expandafter\def\csname PY@tok@gt\endcsname{\def\PY@tc##1{\textcolor[rgb]{0.00,0.27,0.87}{##1}}}
\expandafter\def\csname PY@tok@err\endcsname{\def\PY@bc##1{\setlength{\fboxsep}{0pt}\fcolorbox[rgb]{1.00,0.00,0.00}{1,1,1}{\strut ##1}}}
\expandafter\def\csname PY@tok@kc\endcsname{\let\PY@bf=\textbf\def\PY@tc##1{\textcolor[rgb]{0.00,0.50,0.00}{##1}}}
\expandafter\def\csname PY@tok@kd\endcsname{\let\PY@bf=\textbf\def\PY@tc##1{\textcolor[rgb]{0.00,0.50,0.00}{##1}}}
\expandafter\def\csname PY@tok@kn\endcsname{\let\PY@bf=\textbf\def\PY@tc##1{\textcolor[rgb]{0.00,0.50,0.00}{##1}}}
\expandafter\def\csname PY@tok@kr\endcsname{\let\PY@bf=\textbf\def\PY@tc##1{\textcolor[rgb]{0.00,0.50,0.00}{##1}}}
\expandafter\def\csname PY@tok@bp\endcsname{\def\PY@tc##1{\textcolor[rgb]{0.00,0.50,0.00}{##1}}}
\expandafter\def\csname PY@tok@fm\endcsname{\def\PY@tc##1{\textcolor[rgb]{0.00,0.00,1.00}{##1}}}
\expandafter\def\csname PY@tok@vc\endcsname{\def\PY@tc##1{\textcolor[rgb]{0.10,0.09,0.49}{##1}}}
\expandafter\def\csname PY@tok@vg\endcsname{\def\PY@tc##1{\textcolor[rgb]{0.10,0.09,0.49}{##1}}}
\expandafter\def\csname PY@tok@vi\endcsname{\def\PY@tc##1{\textcolor[rgb]{0.10,0.09,0.49}{##1}}}
\expandafter\def\csname PY@tok@vm\endcsname{\def\PY@tc##1{\textcolor[rgb]{0.10,0.09,0.49}{##1}}}
\expandafter\def\csname PY@tok@sa\endcsname{\def\PY@tc##1{\textcolor[rgb]{0.73,0.13,0.13}{##1}}}
\expandafter\def\csname PY@tok@sb\endcsname{\def\PY@tc##1{\textcolor[rgb]{0.73,0.13,0.13}{##1}}}
\expandafter\def\csname PY@tok@sc\endcsname{\def\PY@tc##1{\textcolor[rgb]{0.73,0.13,0.13}{##1}}}
\expandafter\def\csname PY@tok@dl\endcsname{\def\PY@tc##1{\textcolor[rgb]{0.73,0.13,0.13}{##1}}}
\expandafter\def\csname PY@tok@s2\endcsname{\def\PY@tc##1{\textcolor[rgb]{0.73,0.13,0.13}{##1}}}
\expandafter\def\csname PY@tok@sh\endcsname{\def\PY@tc##1{\textcolor[rgb]{0.73,0.13,0.13}{##1}}}
\expandafter\def\csname PY@tok@s1\endcsname{\def\PY@tc##1{\textcolor[rgb]{0.73,0.13,0.13}{##1}}}
\expandafter\def\csname PY@tok@mb\endcsname{\def\PY@tc##1{\textcolor[rgb]{0.40,0.40,0.40}{##1}}}
\expandafter\def\csname PY@tok@mf\endcsname{\def\PY@tc##1{\textcolor[rgb]{0.40,0.40,0.40}{##1}}}
\expandafter\def\csname PY@tok@mh\endcsname{\def\PY@tc##1{\textcolor[rgb]{0.40,0.40,0.40}{##1}}}
\expandafter\def\csname PY@tok@mi\endcsname{\def\PY@tc##1{\textcolor[rgb]{0.40,0.40,0.40}{##1}}}
\expandafter\def\csname PY@tok@il\endcsname{\def\PY@tc##1{\textcolor[rgb]{0.40,0.40,0.40}{##1}}}
\expandafter\def\csname PY@tok@mo\endcsname{\def\PY@tc##1{\textcolor[rgb]{0.40,0.40,0.40}{##1}}}
\expandafter\def\csname PY@tok@ch\endcsname{\let\PY@it=\textit\def\PY@tc##1{\textcolor[rgb]{0.25,0.50,0.50}{##1}}}
\expandafter\def\csname PY@tok@cm\endcsname{\let\PY@it=\textit\def\PY@tc##1{\textcolor[rgb]{0.25,0.50,0.50}{##1}}}
\expandafter\def\csname PY@tok@cpf\endcsname{\let\PY@it=\textit\def\PY@tc##1{\textcolor[rgb]{0.25,0.50,0.50}{##1}}}
\expandafter\def\csname PY@tok@c1\endcsname{\let\PY@it=\textit\def\PY@tc##1{\textcolor[rgb]{0.25,0.50,0.50}{##1}}}
\expandafter\def\csname PY@tok@cs\endcsname{\let\PY@it=\textit\def\PY@tc##1{\textcolor[rgb]{0.25,0.50,0.50}{##1}}}

\def\PYZbs{\char`\\}
\def\PYZus{\char`\_}
\def\PYZob{\char`\{}
\def\PYZcb{\char`\}}
\def\PYZca{\char`\^}
\def\PYZam{\char`\&}
\def\PYZlt{\char`\<}
\def\PYZgt{\char`\>}
\def\PYZsh{\char`\#}
\def\PYZpc{\char`\%}
\def\PYZdl{\char`\$}
\def\PYZhy{\char`\-}
\def\PYZsq{\char`\'}
\def\PYZdq{\char`\"}
\def\PYZti{\char`\~}
% for compatibility with earlier versions
\def\PYZat{@}
\def\PYZlb{[}
\def\PYZrb{]}
\makeatother


    % Exact colors from NB
    \definecolor{incolor}{rgb}{0.0, 0.0, 0.5}
    \definecolor{outcolor}{rgb}{0.545, 0.0, 0.0}



    
    % Prevent overflowing lines due to hard-to-break entities
    \sloppy 
    % Setup hyperref package
    \hypersetup{
      breaklinks=true,  % so long urls are correctly broken across lines
      colorlinks=true,
      urlcolor=urlcolor,
      linkcolor=linkcolor,
      citecolor=citecolor,
      }
    % Slightly bigger margins than the latex defaults
    
    \geometry{verbose,tmargin=1in,bmargin=1in,lmargin=1in,rmargin=1in}
    
    

    \begin{document}
    
    
    \maketitle
    
    

    
    \begin{Verbatim}[commandchars=\\\{\}]
{\color{incolor}In [{\color{incolor}4}]:} \PY{n}{help}\PY{p}{(}\PYZdq{}\PY{n}{min}\PYZdq{}\PY{p}{)}
        \PY{n}{help}\PY{p}{(}\PYZdq{}\PY{n}{max}\PYZdq{}\PY{p}{)}
        \PY{n}{help}\PY{p}{(}\PYZdq{}\PY{n}{mean}\PYZdq{}\PY{p}{)}
\end{Verbatim}


    \begin{Verbatim}[commandchars=\\\{\}]
 MIN    Smallest component.
    For vectors, MIN(X) is the smallest element in X. For matrices,
    MIN(X) is a row vector containing the minimum element from each
    column. For N-D arrays, MIN(X) operates along the first
    non-singleton dimension.
 
    [Y,I] = MIN(X) returns the indices of the minimum values in vector I.
    If the values along the first non-singleton dimension contain more
    than one minimal element, the index of the first one is returned.
 
    MIN(X,Y) returns an array with the smallest elements taken from X or Y.
    X and Y must have compatible sizes. In the simplest cases, they can be
    the same size or one can be a scalar. Two inputs have compatible sizes
    if, for every dimension, the dimension sizes of the inputs are either
    the same or one of them is 1.
 
    MIN(X,[],'all') returns the smallest element of X.
 
    [Y,I] = MIN(X,[],DIM) operates along the dimension DIM.
 
    MIN(X,[],VECDIM) operates on the dimensions specified in the vector 
    VECDIM. For example, MIN(X,[],[1 2]) operates on the elements contained
    in the first and second dimensions of X.
 
    When X is complex, the minimum is computed using the magnitude
    MIN(ABS(X)). In the case of equal magnitude elements, then the phase
    angle MIN(ANGLE(X)) is used.
 
    MIN({\ldots},NANFLAG) specifies how NaN (Not-A-Number) values are treated.
    NANFLAG can be:
    'omitnan'    - Ignores all NaN values and returns the minimum of the 
                   non-NaN elements.  If all elements are NaN, then the
                   first one is returned.
    'includenan' - Returns NaN if there is any NaN value.  The index points
                   to the first NaN element.
    Default is 'omitnan'.
 
    Example: 
        X = [2 8 4; 7 3 9]
        min(X,[],1)
        min(X,[],2)
        min(X,5)
 
    See also MAX, BOUNDS, CUMMIN, MEDIAN, MEAN, SORT, MINK.

    Reference page in Doc Center
       doc min

    Other functions named min

       categorical/min      duration/min    gpuArray/min    tall/min
       codistributed/min    fints/min       sym/min         timeseries/min
       datetime/min

 MAX    Largest component.
    For vectors, MAX(X) is the largest element in X. For matrices,
    MAX(X) is a row vector containing the maximum element from each
    column. For N-D arrays, MAX(X) operates along the first
    non-singleton dimension.
 
    [Y,I] = MAX(X) returns the indices of the maximum values in vector I.
    If the values along the first non-singleton dimension contain more
    than one maximal element, the index of the first one is returned.
 
    MAX(X,Y) returns an array with the largest elements taken from X or Y.
    X and Y must have compatible sizes. In the simplest cases, they can be
    the same size or one can be a scalar. Two inputs have compatible sizes
    if, for every dimension, the dimension sizes of the inputs are either
    the same or one of them is 1.
 
    MAX(X,[],'all') returns the largest element of X.
 
    [Y,I] = MAX(X,[],DIM) operates along the dimension DIM.
 
    MAX(X,[],VECDIM) operates on the dimensions specified in the vector 
    VECDIM. For example, MAX(X,[],[1 2]) operates on the elements contained
    in the first and second dimensions of X.
 
    When X is complex, the maximum is computed using the magnitude
    MAX(ABS(X)). In the case of equal magnitude elements, then the phase
    angle MAX(ANGLE(X)) is used.
 
    MAX({\ldots},NANFLAG) specifies how NaN (Not-A-Number) values are treated.
    NANFLAG can be:
    'omitnan'    - Ignores all NaN values and returns the maximum of the 
                   non-NaN elements.  If all elements are NaN, then the
                   first one is returned.
    'includenan' - Returns NaN if there is any NaN value.  The index points
                   to the first NaN element.
    Default is 'omitnan'.
 
    Example: 
        X = [2 8 4; 7 3 9]
        max(X,[],1)
        max(X,[],2)
        max(X,5)
 
    See also MIN, BOUNDS, CUMMAX, MEDIAN, MEAN, SORT, MAXK.

    Reference page in Doc Center
       doc max

    Other functions named max

       categorical/max      duration/max    gpuArray/max    tall/max
       codistributed/max    fints/max       sym/max         timeseries/max
       datetime/max

 MEAN   Average or mean value.
    S = MEAN(X) is the mean value of the elements in X if X is a vector. 
    For matrices, S is a row vector containing the mean value of each 
    column. 
    For N-D arrays, S is the mean value of the elements along the first 
    array dimension whose size does not equal 1.
 
    MEAN(X,'all') is the mean of all elements in X.
 
    MEAN(X,DIM) takes the mean along the dimension DIM of X.
 
    MEAN(X,VECDIM) operates on the dimensions specified in the vector 
    VECDIM. For example, MEAN(X,[1 2]) operates on the elements contained
    in the first and second dimensions of X.
 
    S = MEAN({\ldots},TYPE) specifies the type in which the mean is performed, 
    and the type of S. Available options are:
 
    'double'    -  S has class double for any input X
    'native'    -  S has the same class as X
    'default'   -  If X is floating point, that is double or single,
                   S has the same class as X. If X is not floating point, 
                   S has class double.
 
    S = MEAN({\ldots},NANFLAG) specifies how NaN (Not-A-Number) values are 
    treated. The default is 'includenan':
 
    'includenan' - the mean of a vector containing NaN values is also NaN.
    'omitnan'    - the mean of a vector containing NaN values is the mean 
                   of all its non-NaN elements. If all elements are NaN,
                   the result is NaN.
 
    Example:
        X = [1 2 3; 3 3 6; 4 6 8; 4 7 7]
        mean(X,1)
        mean(X,2)
 
    Class support for input X:
       float: double, single
       integer: uint8, int8, uint16, int16, uint32,
                int32, uint64, int64
 
    See also MEDIAN, STD, MIN, MAX, VAR, COV, MODE.

    Reference page in Doc Center
       doc mean

    Other functions named mean

       codistributed/mean    fints/mean       tall/mean
       datetime/mean         gpuArray/mean    timeseries/mean
       duration/mean


    \end{Verbatim}

    \begin{Verbatim}[commandchars=\\\{\}]
{\color{incolor}In [{\color{incolor}5}]:} \PY{n}{a} \PY{p}{=} \PY{p}{[}\PY{l+m+mi}{2}\PY{p}{,} \PY{l+m+mi}{3}\PY{p}{,} \PY{l+m+mi}{4}\PY{p}{,} \PY{l+m+mi}{7}\PY{p}{,} \PY{o}{\PYZhy{}}\PY{l+m+mi}{4}\PY{p}{,} \PY{l+m+mi}{3}\PY{p}{]}\PY{p}{;}
        \PY{n}{min}\PY{p}{(}\PY{n}{a}\PY{p}{)} \PY{c}{\PYZpc{} el mínimo de a}
        \PY{n}{max}\PY{p}{(}\PY{n}{a}\PY{p}{)} \PY{c}{\PYZpc{} el máximo de a}
        \PY{n}{mean}\PY{p}{(}\PY{n}{a}\PY{p}{)} \PY{c}{\PYZpc{} la media de a}
        \PY{n+nb}{length}\PY{p}{(}\PY{n}{a}\PY{p}{)} \PY{c}{\PYZpc{} longitud de a}
        \PY{n}{c} \PY{p}{=} \PY{p}{[}\PY{l+m+mi}{2}\PY{p}{,} \PY{l+m+mi}{3}\PY{p}{;} \PY{l+m+mi}{4}\PY{p}{,} \PY{l+m+mi}{5}\PY{p}{;} \PY{l+m+mi}{7}\PY{p}{,} \PY{l+m+mi}{1}\PY{p}{]}
        \PY{n+nb}{size}\PY{p}{(}\PY{n}{a}\PY{p}{)}
        \PY{p}{[}\PY{n}{numf}\PY{p}{,} \PY{n}{numc}\PY{p}{]} \PY{p}{=} \PY{n+nb}{size}\PY{p}{(}\PY{n}{a}\PY{p}{)}
\end{Verbatim}


    \begin{Verbatim}[commandchars=\\\{\}]

ans =

    -4


ans =

     7


ans =

    2.5000


ans =

     6


c =

     2     3
     4     5
     7     1


ans =

     1     6


numf =

     1


numc =

     6


    \end{Verbatim}

    \hypertarget{soluciuxf3n-de-sistemas-de-ecuaciones-lineales}{%
\subsection{Solución de sistemas de ecuaciones
lineales}\label{soluciuxf3n-de-sistemas-de-ecuaciones-lineales}}

\textbf{Ejemplo:} Encontrar la solución del siguiente sistema lineal

\[\begin{alignat}{9}
2x_1&&\; + \;&& 6x_2   &&\; + \;&& 7x_3  &&\; + \;&& 9x_4 &&\; = \;&& 2 & \\
3x_1 &&\; + \;&& 7x_2  &&\; + \;&& 2x_3 &&\; + \;&& 5x_4 &&\; = \;&& -1 & \\
4x_1 &&\; - \;&& 2x_2  &&\; + \;&& x_3  &&\; + \;&& 6x_4 &&\; = \;&& 3 & \\
x_1&&\; + \;&&  9x_2   &&\; + \;&& 8x_3 &&\; + \;&& 2x_4 &&\; = \;&& 4 & \\
\end{alignat}\]

Forma matricial \[Ax=b\] donde

    \begin{Verbatim}[commandchars=\\\{\}]
{\color{incolor}In [{\color{incolor}6}]:} \PY{n}{A} \PY{p}{=}\PY{p}{[}\PY{l+m+mi}{2} \PY{l+m+mi}{6} \PY{l+m+mi}{7} \PY{l+m+mi}{9}
            \PY{l+m+mi}{3} \PY{l+m+mi}{7} \PY{l+m+mi}{2} \PY{l+m+mi}{5}
            \PY{l+m+mi}{4} \PY{o}{\PYZhy{}}\PY{l+m+mi}{2} \PY{l+m+mi}{1} \PY{l+m+mi}{6}
            \PY{l+m+mi}{1} \PY{l+m+mi}{9} \PY{l+m+mi}{8} \PY{l+m+mi}{2}\PY{p}{]}\PY{p}{;}
        \PY{n}{b} \PY{p}{=}\PY{p}{[}\PY{l+m+mi}{2} \PY{o}{\PYZhy{}}\PY{l+m+mi}{1} \PY{l+m+mi}{3} \PY{l+m+mi}{4}\PY{p}{]}\PY{o}{\PYZsq{}}\PY{p}{;}
        \PY{n}{x} \PY{p}{=} \PY{n}{A}\PY{o}{\PYZbs{}}\PY{n}{b}
\end{Verbatim}


    \begin{Verbatim}[commandchars=\\\{\}]

x =

    0.9557
   -0.5123
    1.0788
   -0.4877


    \end{Verbatim}

    Verificando el resultado anterior:

    \begin{Verbatim}[commandchars=\\\{\}]
{\color{incolor}In [{\color{incolor}7}]:} \PY{n}{A}\PY{o}{*}\PY{n}{x}\PY{o}{\PYZhy{}}\PY{n}{b}
        \PY{n}{norm}\PY{p}{(}\PY{n}{A}\PY{o}{*}\PY{n}{x}\PY{o}{\PYZhy{}}\PY{n}{b}\PY{p}{)} \PY{c}{\PYZpc{} El valor de la norma debe tender a 0.}
\end{Verbatim}


    \begin{Verbatim}[commandchars=\\\{\}]

ans =

   1.0e-14 *

    0.0444
   -0.1776
         0
         0


ans =

   1.8310e-15


    \end{Verbatim}

    \begin{Verbatim}[commandchars=\\\{\}]
{\color{incolor}In [{\color{incolor}8}]:} \PY{n}{help}\PY{p}{(}\PYZdq{}\PY{n}{norm}\PYZdq{}\PY{p}{)}
\end{Verbatim}


    \begin{Verbatim}[commandchars=\\\{\}]
 NORM   Matrix or vector norm.
      NORM(X,2) returns the 2-norm of X.
 
      NORM(X) is the same as NORM(X,2).
 
      NORM(X,1) returns the 1-norm of X.
 
      NORM(X,Inf) returns the infinity norm of X.
 
      NORM(X,'fro') returns the Frobenius norm of X.
 
    In addition, for vectors{\ldots}
 
      NORM(V,P) returns the p-norm of V defined as SUM(ABS(V).\^{}P)\^{}(1/P).
 
      NORM(V,Inf) returns the largest element of ABS(V).
 
      NORM(V,-Inf) returns the smallest element of ABS(V).
 
    By convention, NaN is returned if X or V contains NaNs.
 
    See also COND, RCOND, CONDEST, NORMEST, HYPOT.

    Reference page in Doc Center
       doc norm

    Other functions named norm

       codistributed/norm    DynamicSystem/norm    mfilt/norm    tall/norm
       dfilt/norm            gpuArray/norm         sym/norm


    \end{Verbatim}

    \hypertarget{gruxe1ficas-con-matlab}{%
\subsection{Gráficas con MATLAB}\label{gruxe1ficas-con-matlab}}

    \begin{Verbatim}[commandchars=\\\{\}]
{\color{incolor}In [{\color{incolor}9}]:} \PY{n}{help}\PY{p}{(}\PYZdq{}\PY{n}{plot}\PYZdq{}\PY{p}{)}
\end{Verbatim}


    \begin{Verbatim}[commandchars=\\\{\}]
 PLOT   Linear plot. 
    PLOT(X,Y) plots vector Y versus vector X. If X or Y is a matrix,
    then the vector is plotted versus the rows or columns of the matrix,
    whichever line up.  If X is a scalar and Y is a vector, disconnected
    line objects are created and plotted as discrete points vertically at
    X.
 
    PLOT(Y) plots the columns of Y versus their index.
    If Y is complex, PLOT(Y) is equivalent to PLOT(real(Y),imag(Y)).
    In all other uses of PLOT, the imaginary part is ignored.
 
    Various line types, plot symbols and colors may be obtained with
    PLOT(X,Y,S) where S is a character string made from one element
    from any or all the following 3 columns:
 
           b     blue          .     point              -     solid
           g     green         o     circle             :     dotted
           r     red           x     x-mark             -.    dashdot 
           c     cyan          +     plus               --    dashed   
           m     magenta       *     star             (none)  no line
           y     yellow        s     square
           k     black         d     diamond
           w     white         v     triangle (down)
                               \^{}     triangle (up)
                               <     triangle (left)
                               >     triangle (right)
                               p     pentagram
                               h     hexagram
                          
    For example, PLOT(X,Y,'c+:') plots a cyan dotted line with a plus 
    at each data point; PLOT(X,Y,'bd') plots blue diamond at each data 
    point but does not draw any line.
 
    PLOT(X1,Y1,S1,X2,Y2,S2,X3,Y3,S3,{\ldots}) combines the plots defined by
    the (X,Y,S) triples, where the X's and Y's are vectors or matrices 
    and the S's are strings.  
 
    For example, PLOT(X,Y,'y-',X,Y,'go') plots the data twice, with a
    solid yellow line interpolating green circles at the data points.
 
    The PLOT command, if no color is specified, makes automatic use of
    the colors specified by the axes ColorOrder property.  By default,
    PLOT cycles through the colors in the ColorOrder property.  For
    monochrome systems, PLOT cycles over the axes LineStyleOrder property.
 
    Note that RGB colors in the ColorOrder property may differ from
    similarly-named colors in the (X,Y,S) triples.  For example, the 
    second axes ColorOrder property is medium green with RGB [0 .5 0],
    while PLOT(X,Y,'g') plots a green line with RGB [0 1 0].
 
    If you do not specify a marker type, PLOT uses no marker. 
    If you do not specify a line style, PLOT uses a solid line.
 
    PLOT(AX,{\ldots}) plots into the axes with handle AX.
 
    PLOT returns a column vector of handles to lineseries objects, one
    handle per plotted line. 
 
    The X,Y pairs, or X,Y,S triples, can be followed by 
    parameter/value pairs to specify additional properties 
    of the lines. For example, PLOT(X,Y,'LineWidth',2,'Color',[.6 0 0]) 
    will create a plot with a dark red line width of 2 points.
 
    Example
       x = -pi:pi/10:pi;
       y = tan(sin(x)) - sin(tan(x));
       plot(x,y,'--rs','LineWidth',2,{\ldots}
                       'MarkerEdgeColor','k',{\ldots}
                       'MarkerFaceColor','g',{\ldots}
                       'MarkerSize',10)
 
    See also PLOTTOOLS, SEMILOGX, SEMILOGY, LOGLOG, PLOTYY, PLOT3, GRID,
    TITLE, XLABEL, YLABEL, AXIS, AXES, HOLD, LEGEND, SUBPLOT, SCATTER.

    Reference page in Doc Center
       doc plot

    Other functions named plot

       alphaShape/plot      dspdata/plot         LinearModel/plot
       blm/plot             empiricalblm/plot    mixconjugateblm/plot
       cfit/plot            fints/plot           mixsemiconjugateblm/plot
       channel/plot         graph/plot           phytree/plot
       clustergram/plot     HeatMap/plot         polyshape/plot
       conjugateblm/plot    iddata/plot          semiconjugateblm/plot
       customblm/plot       idnlarx/plot         sfit/plot
       diffuseblm/plot      idnlhw/plot          tall/plot
       digraph/plot         lassoblm/plot        timeseries/plot


    \end{Verbatim}

    Comando \texttt{plot}:

Sintáxis: \texttt{plot(X,\ Y)}

Plotea el vector \texttt{Y} vs el vector \texttt{X}.

Si \texttt{X} o \texttt{Y} es una matriz, entonces el vector es ploteado
vs las filas o columnas de la matriz, dependiendo si el vector es fila o
columna.

Ejercicio: Grafique \(f(t)=t\mathrm{sen}t\,,0\le t\le 6\).

Ejemplo: Para graficar las funciones:
\[f(t)=t\mathrm{sen}t\,,0\le t\le 6, \text{ y } g(t)=t^2,0\le t\le 6.\]

    \begin{Verbatim}[commandchars=\\\{\}]
{\color{incolor}In [{\color{incolor}11}]:} \PY{n}{t\PYZus{}1} \PY{p}{=} \PY{l+m+mi}{0}\PY{p}{:}\PY{l+m+mf}{0.1}\PY{p}{:}\PY{l+m+mi}{6}\PY{p}{;}
         \PY{n}{y} \PY{p}{=} \PY{p}{[}\PY{n}{t\PYZus{}1}\PY{o}{.*}\PY{n+nb}{sin}\PY{p}{(}\PY{n}{t}\PY{p}{)}\PY{p}{;} \PY{n}{t\PYZus{}1}\PY{o}{.\PYZca{}}\PY{l+m+mi}{2}\PY{p}{]}\PY{p}{;}
         \PY{n}{plot}\PY{p}{(}\PY{n}{t\PYZus{}1}\PY{p}{,}\PY{n}{y}\PY{p}{)}
\end{Verbatim}


    \begin{center}
    \adjustimage{max size={0.9\linewidth}{0.9\paperheight}}{output_10_0.png}
    \end{center}
    { \hspace*{\fill} \\}
    
    Lo mismo se obtiene si se escribe

    \begin{Verbatim}[commandchars=\\\{\}]
{\color{incolor}In [{\color{incolor}12}]:} \PY{n}{t\PYZus{}2} \PY{p}{=} \PY{l+m+mi}{0}\PY{p}{:}\PY{l+m+mf}{0.1}\PY{p}{:}\PY{l+m+mi}{6}\PY{p}{;}
         \PY{n}{plot}\PY{p}{(}\PY{n}{t\PYZus{}2}\PY{p}{,}\PY{n}{t\PYZus{}2}\PY{o}{.*}\PY{n+nb}{sin}\PY{p}{(}\PY{n}{t\PYZus{}2}\PY{p}{)}\PY{p}{,}\PY{n}{t\PYZus{}2}\PY{p}{,}\PY{n}{t\PYZus{}2}\PY{o}{.\PYZca{}}\PY{l+m+mi}{2}\PY{p}{)}
\end{Verbatim}


    \begin{center}
    \adjustimage{max size={0.9\linewidth}{0.9\paperheight}}{output_12_0.png}
    \end{center}
    { \hspace*{\fill} \\}
    
    Sintáxis plot (\texttt{X,Y,S}) S es un carácter o cadena (string)
compuesto de uno o de todos los elementos de las siguientes tres
columnas:

\begin{longtable}[]{@{}ccc@{}}
\toprule
Color & Tipo de marca & Tipo de línea\tabularnewline
\midrule
\endhead
\texttt{y} yellow & \texttt{.} punto & \texttt{-} línea
continua\tabularnewline
\texttt{m} magenta & \texttt{o} círculo & \texttt{:} línea
punteada\tabularnewline
\texttt{c} cyan & \texttt{x} marca-x & \texttt{-.} línea
raya-punto.\tabularnewline
\texttt{r} red & \texttt{+} más & \texttt{-\/-} línea
rayada\tabularnewline
\texttt{g} green & \texttt{*} estrella &\tabularnewline
\texttt{b} blue & s cuadrado &\tabularnewline
\bottomrule
\end{longtable}

    \begin{Verbatim}[commandchars=\\\{\}]
{\color{incolor}In [{\color{incolor} }]:} \PY{n}{El} \PY{n}{objetivo} \PY{n}{de} \PY{n}{los} \PY{n}{métodos} \PY{n}{numéricos} \PY{n}{es} \PY{n}{hallar} \PY{n}{soluciones} \PY{n}{de} \PY{n}{ecuaciones} \PY{n}{desconocidas}\PY{p}{.}\PY{n}{a}
        \PY{n}{Implementar} \PY{n}{la} \PY{n}{matriz} \PY{n}{de} \PY{n}{Vandermonde} \PY{n}{en} \PY{n}{MATLAB}
\end{Verbatim}



    % Add a bibliography block to the postdoc
    
    
    
    \end{document}
